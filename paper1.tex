\hypertarget{paper-1}{%
\chapter{Digital Trace Data as Indicators of Social Data Validating Google Trends for  use in Scientific Research}\label{paper-1}}

Possible titles Digital Trace Data as Social Indicators: Indicators of
Attention Rather Than Support

Digital Trace Data as Indicators of Social Data: Validating Google Trends for
 use in Scientific Research


\section{Abstract}

\section{Introduction}
New big data sources have led to vast possibilities for social science
research because they are bigger, cheaper, and already available
\citep{kingEnsuringDataRichFuture2011,lazerComputationalSocialScience2009,salganikBitBitSocial2017}. Before overenthusiastically embracing these
sources into our workflows, social scientists must clearly establish
parameters under which these data sources could be operationalized
\citep{bailCulturalEnvironmentMeasuring2014, lazerParableGoogleFlu2014}. As
prior research outlined, the "quantity of data does not mean that one
can ignore foundational issues of measurement and construct validity and
reliability and dependencies among data" \citep[p. 1203]{lazerParableGoogleFlu2014}. Building on this prior research outlining these various
issues with big data \citep{boydCriticalQuestionsBig2012,lazerIssuesConstructValidity2015}, this paper tests the construct
validity of Google Search Trends as an indicator of three different
cases, namely cultural attitudes, disease prevalence and voting
behavior. These three cases will be tested using cultural indicators
from the NORC General Social Survey, United States county-level suicide
rates from National Center for Health Statistics, US cases rates of
Covid-19 from The New York Times, historical US Presidential Election
results, and the American National Election Survey. Data will be
analyzed against corresponding operationalized Google Trends
longitudinal data using Pearson's r for pairwise correlations, testing
for the strength of relationships between the Google Trend indicator and
the respective comparison indicator. This paper will contribute to the
creation of methodological norms and standards of how to use Google
Trends as a big data source for societal research and serve as a
critical inquiry into the adoption of big data without a critical eye
for the ecological validity of the sources.

With the expansion of big data, some research has shown extremely
innovative methods that lead to groundbreaking results that are shown to
be reliable. As an example, \citet{blumenstockPredictingPovertyWealth2015} use
county-level cell-phone records to construct the distribution of poverty
and wealth in Rwanda, a country where national surveys and censes are
rare and costly. However, \citet{blumenstockPredictingPovertyWealth2015} go to
great lengths to demonstrate that their operationalization of the cell
phone data creates a reliable and valid construct; few social science
papers utilizing big data dig into the construct validity of their
metrics to this extent and even fewer publications focus on
methodological guidelines of how to use sources of big data
\citep{asseoTrackingCOVID19Using2020, stilesAssessingCriterionValidity2018}.
However, research has shown the small adjustments to an algorithm or
metric may void any research insight we are able to pull from such data
\citep{lazerParableGoogleFlu2014}. Because of this, I propose a
methodological validation study of the Google Search Trends source of
big data to investigate how it is advisable to utilize this data in
social scientific research.

I will use three categorizations of ways I propose Google Trends could
be operationalized for social scientific usage. First, I'll test Google
Trends as an operationalization of cultural attitudes with the General
Social Survey. After \citet{bailCulturalEnvironmentMeasuring2014}'s call for
cultural sociologists to utilize the ever-expanding world of big data,
Google trends as a data source began appearing in sociological and
social science research. From research on mass shootings and firearms
\citep{brownsteinInternetSearchPatterns2020,semenzaInformationseekingWakeTragedy2020}, protest and anti-Muslim sentiment \citep{bailUsingInternetSearch2018,barrieSearchingRacismGeorge2020,grossThereFergusonEffect2017}, to analyzing country-level changes in social perception \citep{reyes_etal18},
Google search trends are a new and innovative indicator of cultural
interest. Extending into social networks and culture, \citet{bailPrestigeProximityPrejudice2019} even used Google trends to measure
how culture spreads around the globe.

Google Search Trends have also been used continuously in estimations of
disease prevalence and population health in journals like the Journal of
Medical Internet Research. While much of this research has focused on
the Covid-19 pandemic \citep{jimenez_etal20,jimenezCOVID19SymptomGoogle2020,limEstimatingInformationSeekingBehaviour2020,mavraganiCOVID19PredictabilityUnited2020,nguyenGoogleTrendsAnalysis2020, todorovaInternetBasedData2021} , other
research has investigated Google Trends as an indicator of wellbeing
\citep{brodeurCOVID19LockdownsWellbeing2021,carpiTwitterSubjectiveWellBeing2020, duCOVID19IncreasesOnline2020},
suicidality \citep{burnettTimeTrendsPublic2020}, vaccination uptake
\citep{dalumhansenEnsembleLearnedVaccination2016}, obesity
\citep{sarigulNowcastingObesityUsing2014}, and even insomnia
\citep{zittingGoogleTrendsReveal2020}, to cover a few examples. For a partial
review of other utilizations, see \citet{nutiUseGoogleTrends2014}. According to
\citet{jaidkaInformationseekingVsSharing2021}, the majority of studies profess
a correlation of \> .70, "demonstrating the vast potential of Google
Search as a proxy for monitoring population health" (p. 3) based on
assumptions that individuals search because of self-diagnosis and to
identify possible courses of treatment
\citep{dechoudhurySeekingSharingHealth2014}.

% <!-- Following reports say that Google searches can predict covid  -->

% <!-- 14. Brunori, P. & Resce, G. Searching for the peak Google Trends and the Covid-19 outbreak in Italy. SERIES working papers N. 04/2020 -->

% <!-- Lampos, V. et al. Tracking COVID-19 using online search. Preprint at https://arxiv.org/abs/2003.08086 (2020). -->

% <!-- 16. Walker, A., Hopkins, C. & Surda, P. The use of google trends to investigate the loss of smell related searches during COVID‐19 outbreak. Int. Forum Allergy Rh. 10, 839–847 (2020). -->

% <!-- 17. Stephens-Davidowitz, S. Google Searches Can Help Us Find Emerging Covid-19 Outbreaks. The New York Times https://www.nytimes.com/2020/04/05/opinion/coronavirus-google-searches.html (2020). -->

% <!-- 18. Frank, T. Goldman says fewer ‘loss of smell’ Google queries suggest better COVID outlook. Consumer News and Business Channel. CNBC https://www.cnbc.com/2020/04/15/goldman-says-fewer-loss-of-smell-google-searches-is-a-positive-sign-for-pandemic.html (2020) -->

Various sources have also used Google Trends as a way to forecast
political elections and political attitudes
\citep{wolfTrendingRightDirection2018}. For instance,
\citet{swearingenGoogleInsightsSenate2014} investigate how U.S. Senate
Elections relate to attention measured by search traffic.
\citet{prado-romanGoogleTrendsPredictor2020} compare how Google Search trends
are able to predict presidential election results in both the United
States and Canada. Finally, the OECD Development Centre is investigating
how Google data can help elucidate governments' approval in Latin
America \citep{montoyaUsingGoogleData2020}.

Research Question - How can we operationalize Google Search Trends as a
valid indicator for uses in social science research?

<https://journals.sagepub.com/doi/10.1177/0894439316631043> What
constructs might google trends capture and not capture well? Capture
attention but not attitudes

Also Asseo - "we assumed that media coverage may potentially decouple
the search popularity from the number of cases, since searches would
result not only from self-symptoms, but also from interest elicited by
media coverage."

\section{Methods}
To investigate the construct and criterion validity of the use of Google Trends
in these three areas, I gathered geo-located social science data across multiple
sources to address the three areas of inquiry for this paper.  Table
\ref{tab:data-sources-table} outlines which data sources are used for this
project and which trends are matched to each source.

\newcolumntype{L}{>{\raggedright\arraybackslash}X}
	
\definecolor{Gray}{gray}{0.9}
\definecolor{lightblue}{HTML}{bad7db}

\renewcommand{\arraystretch}{1.2}

\begin{table}[!ht]
\caption{\label{tab:data-sources-table}Data Sources}
\centering
\fontsize{8}{10}\selectfont
\setlength{\extrarowheight}{2pt}
\begin{tabularx}{\linewidth}{@{}L c c L@{}}
\toprule
\textbf{Validated Data Source} & \textbf{Type} & \textbf{Dates} & \textbf{Google Trends Used}\citep{googletrends} \\ \midrule
% \rowcolor{lightblue}
\multicolumn{4}{l}{\textbf{Behaviors and Attitudes}} \\\hline
% General Social Survey & Cross-Sectional & 2010 - 2020 & NA \\
Vaccine Hesitancy for COVID-19 \citep{vaches_data}& Cross-Sectional & March 3 - 15, 2021 & Search Topics: 'Covid-19 vaccine', 'Coronavirus (Disease)', 'Coronavirus   (Virus)', 'Vaccine' \\
Mask-Wearing Survey Data \citep{mask_data} & Cross-Sectional & July 2 - 14, 2020 & Search Topics: 'Coronavirus (Disease)', 'Coronavirus (Virus)', 'Cloth   Face Mask', 'Mask', 'Civil and Political Rights' \\
American National Election Survey \citep{anes_data}& Cross-Sectional & 2020 & NA \\
% \rowcolor{lightblue}
\multicolumn{4}{l}{\textbf{Health}} \\\hline
Covid Rates \citep{covid_data}& Longitudinal & Every Monday, 2020 - 2021 & 'Covid-19', 'Coronavirus', 'Taste Loss', 'Smell Loss' \\
Suicide Rates \citep{suic_data} & Longitudinal & Yearly 2010-2020 & Search Topics: 'Suicide', 'Depression', Search Term:' Suicide Hotline' \\
% \rowcolor{lightblue}
\multicolumn{4}{l}{\textbf{Political}} \\\hline
American National Election Survey \citep{anes_data}& Cross-Sectional & 2020 & NA \\
Presidential Election Results \citep{pres_data}& Cross Sectional & 2016 \& 2020 & Search Topics: 'Hilary Clinton', 'Donald Trump', 'Joe Biden' \\ \bottomrule
\end{tabularx}
\end{table}










\subsection{Measures}

\subsubsection{Google Trends}
This paper focuses on a validation of Google search trends \citep{googletrends}.
Google trends portray the search frequency for specific search terms across
designated media markets areas (DMAs), a nonoverlapping aggregation of U.S.
counties to 210 media markets based on similar population clusters \citep{dma_key}.
Raw data is on a scale from 0 to 100, with 100 being the maximum search
popularity out of all DMAs. When available, I use Google search topics to
measure trends as opposed to search terms. Search topics are a more robust
measurement than a single search term: topics are aggregations of the rates of
multiple, highly correlated search terms together into a cohesive topic. For
example, while 'Beyoncé', 'Beyonce' and 'beyonce knowles' are all separate
search terms, 'Beyoncé Knowles' encompasses all of these into a single search
topic. Google Search Trend are only available cross-sectionally (a single time
period across a geography) or as time-series (a single geo-location across
time). To remedy this and build a longitudinal dataset of each search topic for
the longitudinal datasets, I follow the method proposed in \citet[p. 5]{park_etal}.
This method involves building a dataset of unscaled cross-sectional values,
selecting a DMA to use to establish the rescaling ratio (I use 'Los Angeles
CA'), and then finding the time-series values for the one DMA. To find the
rescaling ratio for each week in the time-series, you divide the time-series
value for each week by the cross-sectional value for each week, resulting in a
rescaling vector to be used for all weeks in the dataset across geographies. To
rescale each longitudinal value, multiply the respective week's rescaling ratio
by the cross-sectional value. Rescaled longitudinal data was compared against
time-series data for multiple test counties and was equivalent. For a more
in-depth explanation of this procedure, see \citet[p. 5]{park_etal}. Missing
datapoints in longitudinal datasets were filled in with interpolated values
using `zoo::na.approx()` \citep{zoo}.

\subsubsection{Attitudinal}
<!-- \citep{gss_data} (let's see if I get access first) -->

One measure of attitudinal indicators is the Vaccine Hesitancy for COVID-19
\citep{vaches_data}. The CDC uses the U.S. Census Bureau’s Household Pulse Survey
(HPS) and the 2019 American Community Survey (ACS) 1-year Public Use Microdata
Sample (PUMS) to measure U.S. residents’ intentions to receive the COVID-19
vaccine if available during May 26, 2021 – June 7, 2021. This dataset consists
of x observations, one for each U.S. County. The variable
measures the percent of adults in the county who describe themselves as
“unsure”, “probably not”, or “definitely not” going to get a COVID-19 vaccine
once one is available to them. The variable ranges from x to y. 

Another attitudinal indicator I use is the Mask-Wearing Survey Data conducted by
Dynata for the New York Times from July 2 through 14, 2020 \citep{mask_data}. 250,000
survey respondents were asked, "How often do you wear a mask in public when you
expect to be within six feet of another person?". The NYT weighted each response
to create a county level measure of what percent of the county never, rarely,
sometimes, frequently, and always wore a mask when in public. This dataset
consists of x observations, one for each U.S. County. This measure
represents the percent of adults in the county who never or rarely wear a mask
(range = x to x).

\subsubsection{Health}


I also use U.S. Covid-19 rates to validate health and disease related topics. I
retrieve U.S. county-level Covid-19 rates from by \citet{covid_data}, who compile this
data based on reports from state and local health agencies. It is widely
acknowledged that there are biases in this data due to inconsistencies and
availability in testing as well as different community propensity to test
\citep{gu22, cdc20a} However, it is the best measure we have of actual case rates.
Case Rate is measured as number of cases per 100,000 population. Observations
vary from a minimum of 0 to a maximum of x for each 
Monday from x through x. There are x cases across x counties and x dates. Missing data were interpolated using `zoo::na.approx()` \citep{zoo}.

I also use county-level suicide rates from the US Centers for Disease Control
and Prevention \citeyearpar{suic_data}. Data is grouped by year from 2010-2020. Raw death
rates are scaled by population size for each year and can be interpreted as the
death rate by suicide for every 1000 people. There are x total
cases, resulting from x observations of x counties. Missing data were interpolated using
zoo::na.approx() \citep{zoo}. Measures range from x to x.

\subsubsection{Political}

Finally, I test Google Trends as an operationalization of political
attitudes by first looking at actual voting outcomes in historical US
Presidential Election results. This data comes from  \citet{pres_data}, who scraped the
results from Townhall.com, Fox News, Politico, and the New York Times. Data on 
presidential outcomes were available for x counties in 2016 and
x counties in 2020. Each variable measures the percent of votes
for the candidate, with the lowest percent at x and the highest at x. 

In addition, I use data on political opinions from the the American National
Election Survey 2020 Time-Series Study \citep{anes_data}. This data is paired with
restricted data provided by ICPSR, the Inter-university Consortium for Political
and Social Research, that contains the geo-ids of the 5,441
survey respondents. The study interviewed respondents in a pre-election survey
that was conducted between August 18, 2020 and the day of the US Presidential
election day, November 3, 2020.

% TODO: WHICH VARIABLES
% TODO: VARIABLE SCALE

\subsubsection{Other}

In addition to specific outcomes of interest, I also gathered various county
controls to investigate possible variable confounding. The first Seven of these
variables come from the 2010-2019 5-year American Community Surveys
\citep{acs2019, acs2018, acs2017, acs2016, acs2015, acs2014, acs2013, acs2012, acs2011, acs2010}. 
These data include total population, population density,
unemployment rate for those 16 and over, median county income, average commute
time, percent of households living under the poverty line, and percent above 65
years old. In addition, some models employ the U.S. Current Population Survey \&
American Community Survey Geographic Estimates of Internet Use, 1997-2018
\citep{internet_use} to estimate households with broadband internet subscriptions.
This final variable is an attempt to capture the latent propensity to use the
internet for information search.

% ACS 5 year estimates
% a00001 total population
% a00002 population density
% a17005 unemployment rate 16+
% a14006 median income
% a09003 average commute time
% poverty percent
% 01001B 65 plus

\subsection{Analysis}
Google Search Trends data and additional demographic data are merged with each
individual indicator based on County FIPS Codes and date. After creating these
different datasets, I use the Pearson correlation formula (formula \eqref{eq:pearsoncorr}
) for cross-sectional numeric data to calculate the
strength of the relationship between each Google Trend and the respective
data source.

\begin{equation}
 r =
  \frac{ \sum_{i=1}^{n}(x_i-\bar{x})(y_i-\bar{y}) }{
        \sqrt{\sum_{i=1}^{n}(x_i-\bar{x})^2}\sqrt{\sum_{i=1}^{n}(y_i-\bar{y})^2}} \label{eq:pearsoncorr}
\end{equation}

To address longitudinal correlations, I employ Repeated Measures Correlation
using the `rmcorr` package in R \citep{bland1995, bakdash2017}. Repeated Measures
Correlation is useful for determining the within-county association for paired
measures across time and counties. Results of these correlations can be seen in 
table \ref{tab:corr-results}.

As an additional test of the relationships, I employ multiple linear regression
(for cross-sectional datasets) and random intercept hierarchical linear models
\citep{pinheiro_etal21} (for longitudinal datasets) to identify the strength of
relationships across locales: I include city-data like county population size,
broadband rates, and median income to attempt to disentangle possible
confounders of the relationship between Google search Trends and verified
indicators. Identifying these possible confounded relationships will help to
explain why some articles find relationships between the trends and outcomes
while others did not. For linear regression models, I normalize independent
variables, i.e.variables have been centered and scaled to have a mean of 0 and
standard deviation 1.

\section{Results}

\subsection{Validating Google Trends as metrics of attitudinal indicators}

% \citep{gss_data} (let's see if I get access first)

\begin{table}[!h]
\caption{\label{tab:corr-results}}
\centering
\begin{tabular}[t]{}
\end{tabular}
\end{table}

The first attitudinal indicators is Vaccine Hesitancy for COVID-19 vaccines
\citep{vaches_data}. When comparing the measure of vaccine hesitancy to four
different Google Search Trends, the correlation does not exceed -0.4657
('Coronavirus (Disease)' and vaccine hesitancy). Correlation's under |0.40| are
considered to be weak according to the common rules of thumb. When running these
correlations in multiple linear regression (see the results in table
\ref{tab:vacc_hes_analysis}), I see an $r^2$ of 0.233 (Model 1) and 0.411
(Model 2), indicating that the Google Trends are able to explain about 23% of
the variation in Vaccine Hesitancy alone. Demographic characteristics like the
percentage of households with broadband internet and the population density are
able to explain about 35\% of the variation (model 2), outperforming the first
models. The Trend coefficients themselves, however, are significant and remain
significant when controlling for demographics. This reinforces the finding from
the Pearson Correlation that there is a significant but weak relationship
between the Google Trends and Vaccine Hesitancy.

\begin{table}[!h]
\caption{\label{tab:vacc_hes_analysis}}
\centering
\begin{tabular}[t]{}
\end{tabular}
\end{table}

The second attitudinal measure I test is how Google Trends relates to rare mask
usage. As with vaccine hesitancy, the Pearson correlations are negligible.
Correlation's under |0.20| are considered to be negigible according to the
common rules of thumb. I introduce these trends in multiple linear regression in
table \ref{tab:mask_analysis}. Model 1 demonstrates that these five Google
Search Trends can explain about 7\% of the variance in mask usage across U.S.
counties, reinforcing the conclusion that the relationship is quite weak. The
coefficients themselves are somewhat significant in Model 1. However, after
including the demographic variables in Models 2 and 3, we see that the
relationship between the Google Search Trends and mask usage is strengthened in
magnitude and in significance, likely indicating a suppression effect due to
underlying relationships between the trends and demographic variables. While the
trends are significant and match the demographic variables in magnitude, the
low $r^2$ of 0.203 for the trends still provides evidence that Google Search
Trends data cannot replace survey analysis when trying to measure rare mask
usage.

\begin{table}[!h]

\caption{\label{tab:mask_analysis}Linear Regression Results for Rare Mask Usage}
\centering
\fontsize{8}{10}\selectfont

\begin{tabular}{lccc}
\toprule
  & Model 1 & Model 2 & Model 3\\
\midrule

Search for `Coronavirus Disease' & \num{-0.010}*** &  & \num{-0.012}***\\
 & (\num{0.003}) &  & \vphantom{1} (\num{0.003})\\
Search for `Coronavirus Virus' & \num{-0.012}*** &  & \num{-0.001}\\
 & (\num{0.003}) &  & (\num{0.003})\\
Search for `Cloth face mask' & \num{-0.009}* &  & \num{-0.015}***\\
 & (\num{0.004}) &  & (\num{0.004})\\
Search for `Mask' & \num{0.025}*** &  & \num{0.023}***\\
 & (\num{0.004}) &  & (\num{0.003})\\
Search for `Civil and political rights' & \num{-0.012}*** &  & \num{-0.008}***\\
 & (\num{0.002}) &  & (\num{0.002})\\
Total Population &  & \num{-0.012}*** & \num{-0.011}***\\
 &  & (\num{0.002}) & \vphantom{3} (\num{0.002})\\
Population Density &  & \num{-0.003}+ & \num{-0.004}*\\
 &  & (\num{0.002}) & \vphantom{2} (\num{0.002})\\
Unemployment Rate &  & \num{-0.028}*** & \num{-0.024}***\\
 &  & (\num{0.002}) & \vphantom{1} (\num{0.002})\\
\% over 65 &  & \num{-0.006}** & \num{-0.003}+\\
 &  & (\num{0.002}) & (\num{0.002})\\
\% below poverty line &  & \num{-0.010}** & \num{-0.009}**\\
 &  & (\num{0.003}) & \vphantom{2} (\num{0.003})\\
Median income &  & \num{-0.036}*** & \num{-0.030}***\\
 &  & (\num{0.003}) & \vphantom{1} (\num{0.003})\\
\% with broadband &  & \num{-0.006}* & \num{-0.007}**\\
 &  & (\num{0.003}) & (\num{0.003})\\
\midrule
Num.Obs. & \num{2964} & \num{3133} & \num{2954}\\
R2 & \num{0.077} & \num{0.176} & \num{0.203}\\
R2 Adj. & \num{0.076} & \num{0.174} & \num{0.200}\\
\bottomrule
\multicolumn{4}{l}{\rule{0pt}{1em}* p $<$ .05. ** p $<$ .01. *** p $<$ .001 (two-tailed test).}\\
\end{tabular}
\end{table}

\subsection{Validating Google Trends as metrics of health outcomes}

In addition to attitudinal measures, I attempt to validate Google Trends for
uses in the measurements of health indicators. The first indicator I test is
Covid-19 case rates from 2020 through 2021. The Pearson correlation results
indicate negligible to weak relationships between the search terms and the
actual case rates across time and place. The repeated measures correlation
coefficient, a more reliable and less biased measure for longitudinal
correlations, reveals similarly weak results with a maximum correlation of 0.32
between 'smell loss' and rates of Covid-19. To further test the relationship, I
include the three trends in random intercept hierarchical linear models. Model 1
of table \ref{tab:covid_analysis} indicates that the trends themselves are able
to explain about 20\% of the variation in Covid-19 rates ($r^2$ = 0.201). Few
of the demographic variables have an effect on Covid-19 rates in models 2 or 3,
though it seems there is some suppression for median income, unemployment rates,
and population density. Model 3 has an $r^2$ of 0.200. I conclude that the
Google Trends are slightly related to Covid-19 rates, but that this relationship
is weak and we should not attempt to use Google Trends as an indicator of actual
health outcomes.

\begin{table}[!h]

\caption{\label{tab:covid_analysis}Hierarchical Model for Covid Case Rates}
\centering
\fontsize{8}{10}\selectfont

\begin{tabular}{lccc}
\toprule
  & Model 1 & Model 2 & Model 3\\
\midrule
covid\_19 & \num{1.031}*** &  & \num{1.024}***\\
 & (\num{0.082}) &  & \vphantom{1} (\num{0.082})\\
smell\_loss & \num{7.373}*** &  & \num{7.374}***\\
 & (\num{0.082}) &  & (\num{0.082})\\
taste\_loss & \num{6.344}*** &  & \num{6.347}***\\
 & (\num{0.078}) &  & (\num{0.078})\\
covid\_rate\_fips\_mean & \num{0.813}*** & \num{0.982}*** & \num{0.879}***\\
 & (\num{0.019}) & (\num{0.010}) & (\num{0.020})\\
date & \num{0.029}*** & \num{0.038}*** & \num{0.029}***\\
 & (\num{0.000}) & (\num{0.000}) & (\num{0.000})\\
total\_pop &  & \num{0.064} & \num{0.409}**\\
 &  & (\num{0.066}) & (\num{0.140})\\
pop\_density &  & \num{0.010} & \num{0.463}***\\
 &  & (\num{0.067}) & (\num{0.123})\\
unemployment\_rate &  & \num{0.049} & \num{0.725}***\\
 &  & (\num{0.080}) & (\num{0.155})\\
over\_65 &  & \num{-0.116} & \num{-0.055}\\
 &  & (\num{0.074}) & (\num{0.139})\\
poverty\_rate &  & \num{0.054} & \num{0.351}\\
 &  & (\num{0.118}) & (\num{0.221})\\
median\_income &  & \num{0.031} & \num{0.907}***\\
 &  & (\num{0.114}) & (\num{0.214})\\
broadband &  & \num{0.073} & \num{0.437}*\\
 &  & (\num{0.097}) & (\num{0.182})\\
\midrule
Num.Obs. & \num{248346} & \num{284666} & \num{247655}\\
R2 Marg. & \num{0.201} & \num{0.081} & \num{0.200}\\
R2 Cond. &  & \num{0.081} & \\
\bottomrule
\multicolumn{4}{l}{\rule{0pt}{1em}Random intercept per county}\\
\multicolumn{4}{l}{\rule{0pt}{1em}* p $<$ .05. ** p $<$ .01. *** p $<$ .001 (two-tailed test).}\\
\end{tabular}
\end{table}

Another indicator of health outcomes I test to validate Google Trends as
possible health indicators are county suicide rates. As with previous cases,
there is only negligible correlation between actual suicide rates and Google
Search Trends, not exceeding 0.0845. In the more accurate repeated measures
correlation coefficient we see a maximum of 0.142 correlation, remaining
negligible. I introduce these trends into a random intercept model to control
for inter-group variation over time in table \ref{tab:suicide_analysis}. While
table \ref{tab:suicide_analysis} model 1 reveals a significant relationship
between suicide hotline trends and suicide rates, the effect is small and the
entire model only has an $r^2$ of 0.058. Adding in demographic features in model
2 improves the $r^2$ quite a lot. In model 3, the inclusion of Google Trends
actually decreases the overall amount of suicide rate variance explained
compared to model 2. These analyses lead me to conclude that Google Search
Trends should not be used as indicators of suicide rates or intentions.

\begin{table}[!h]

\caption{\label{tab:suicide_analysis}Fixed Effect Model for Suicide Rates}
\centering
\fontsize{8}{10}\selectfont

\begin{tabular}{lccc}
\toprule
  & Model 1 & Model 2 & Model 3\\
\midrule

Search for `Suicide' & \num{-0.001} &  & \num{-0.001}\\
 & (\num{0.001}) &  & \vphantom{2} (\num{0.001})\\
Search for `Depression' & \num{-0.001}+ &  & \num{-0.001}\\
 & (\num{0.001}) &  & \vphantom{1} (\num{0.001})\\
Search for `Suicide Hotline' & \num{0.002}*** &  & \num{0.002}***\\
 & (\num{0.001}) &  & (\num{0.001})\\
Year & \num{0.004}*** & \num{0.005}*** & \num{0.005}***\\
 & (\num{0.000}) & (\num{0.000}) & (\num{0.000})\\
Total Population &  & \num{-0.032}*** & \num{-0.027}**\\
 &  & (\num{0.010}) & (\num{0.009})\\
Population Density &  & \num{-0.037}** & \num{-0.033}**\\
 &  & (\num{0.011}) & (\num{0.011})\\
Unemployment Rate &  & \num{-0.001} & \num{-0.001}\\
 &  & (\num{0.001}) & \vphantom{1} (\num{0.001})\\
\% over 65 &  & \num{-0.010}*** & \num{-0.014}***\\
 &  & (\num{0.002}) & \vphantom{1} (\num{0.002})\\
\% below poverty line &  & \num{-0.005}*** & \num{-0.003}**\\
 &  & (\num{0.001}) & (\num{0.001})\\
Median income &  & \num{-0.014}*** & \num{-0.015}***\\
 &  & (\num{0.002}) & (\num{0.002})\\
\midrule
Num.Obs. & \num{30008} & \num{33582} & \num{30001}\\
R2 Marg. & \num{0.058} & \num{0.582} & \num{0.564}\\
\bottomrule
\multicolumn{4}{l}{\rule{0pt}{1em}Fixed intercept per county}\\
\multicolumn{4}{l}{\rule{0pt}{1em}* p $<$ .05. ** p $<$ .01. *** p $<$ .001 (two-tailed test).}\\
\end{tabular}
\end{table}

\subsection{Validating Google Trends as metrics of political support}

Previous research has showed some relationship between Google Search Trends and
political election results. I test here the relationships between search trends
and U.S. Presidential Election results in 2016 and 2020. In 2016, Pearson
correlations for Both Hilary Clinton and Donald Trump do not exceed a |0.17|
correlation with the actual percentage of votes for either candidate. In 2020,
results are even less related, with searches for neither Joe Biden nor Donald
Trump holding any real correlation with the 2020 results.

In table \ref{tab:pres_2016_analysis}, I outline the results for the model
predicting the county percentage of votes for Hillary Clinton. Model 1
demonstrates how well Google Trends are able to predict the votes; while the
trend for 'Hillary Clinton' is significantly associated with votes, the model's
$r^2$ ($r^2$ = 0.027) shows that it does little to help explain the model
variance. On the other hand, adding in demographic features improves the model
fit quite well, bringing the $r^2$ up to 0.313 in model 2 and $r^2$ 0.327 in
model 3. In table \ref{tab:pres_2020_analysis} shows this same analysis for the
2020 election and predicts the percentage of votes for Joe Biden. While this
model is largely similar to those in table \ref{tab:pres_2016_analysis}, this
model 1 demonstrates how unrelated Google Trends can be from actual outcomes. In
conclusion, I find that Google Trends are an unreliable and invalid indicator of
political support.

\begin{table}[!h]
\caption{\label{tab:pres_2016_analysis}}
\centering
\begin{tabular}[t]{}
\end{tabular}
\end{table}
\begin{table}[!h]

\caption{\label{tab:pres_2020_analysis}Linear Regression Results for 2020 Presidential Election Results (Joe Biden Shown)}
\centering
\fontsize{8}{10}\selectfont

\begin{tabular}{lccc}
\toprule
  & Model 1 & Model 2 & Model 3\\
\midrule

Search for `Joe Biden' & \num{-0.005} &  & \num{-0.014}***\\
 & (\num{0.003}) &  & (\num{0.002})\\
Total Population &  & \num{0.030}*** & \num{0.029}***\\
 &  & (\num{0.003}) & \vphantom{3} (\num{0.003})\\
Population Density &  & \num{0.021}*** & \num{0.022}***\\
 &  & (\num{0.003}) & \vphantom{2} (\num{0.003})\\
Unemployment Rate &  & \num{0.044}*** & \num{0.044}***\\
 &  & (\num{0.003}) & \vphantom{1} (\num{0.003})\\
\% over 65 &  & \num{-0.007}* & \num{-0.008}**\\
 &  & (\num{0.003}) & (\num{0.003})\\
\% below poverty line &  & \num{0.032}*** & \num{0.033}***\\
 &  & (\num{0.004}) & \vphantom{1} (\num{0.004})\\
Median income &  & \num{0.069}*** & \num{0.070}***\\
 &  & (\num{0.004}) & (\num{0.004})\\
\midrule
Num.Obs. & \num{3118} & \num{3118} & \num{3118}\\
R2 & \num{0.001} & \num{0.295} & \num{0.303}\\
R2 Adj. & \num{0.000} & \num{0.294} & \num{0.301}\\
\bottomrule
\multicolumn{4}{l}{\rule{0pt}{1em}Results predicting Donald J. Trump comparable and available upon request.}\\
\multicolumn{4}{l}{\rule{0pt}{1em}* p $<$ .05. ** p $<$ .01. *** p $<$ .001 (two-tailed test).}\\
\end{tabular}
\end{table}

\section{Discussion}

The external validity of these voting 


While I expect these tests to show high correlation between observed
indicators and Google search trends, there will be three important
questions that surface. First, just because something is correlated,
does that mean it can replace the collection of other types of data?
Second, how correlated does a trend need to be for social scientists to
justifiably rely on it to indicate some outcome? And finally, how can we
construct analyses like this to be robust to changes in the terms used
across time and location?

The purpose of this paper is more methodological than theoretical, and I
see this paper having an impact on the social sciences and computational
social science as researchers pursue more projects using this source of
big data. Google Trends are relatively underutilized in the field
compared to in the health sciences and business. Once I assess how this
data can be used, I would like to be able to join
\citet{bailCulturalEnvironmentMeasuring2014} in encouraging social scientists
to pursue more research with big data while taking into account the
potential pitfalls with any source of big data
\citep{mcfarlandBigDataDanger2015}.

\section{Conclusion}