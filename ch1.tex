\hypertarget{intro}{%
\chapter{Introduction}\label{intro}}

% TODO entire dissertation Covid vs COVID vs Coronavirus


This dissertation focuses on how people search for information and how people
rely on this information and to inform their health-behaviors and develop social
norms. Information permeates every facet of human society, from the hordes of
data gathered through digital trace data or the passing gossip between friends.
Human communication and human society are based on the circulation of
information and knowledge. Information can be shared and promoted by others
through information campaigns and consumed by an individual, called information
push, or intentionally sought out by individuals, called information pull
\citep{cybenkoFoundationsInformationPush1999}.

We must understand the information individuals receive and seek out in order
to develop a well functioning society. The information that
one consumes consolidates to create the entire worldview of every individual,
leading to how they understand the world and behave civilly therein. New and
emerging social norms must align with one's worldview to anticipate any sort of 
behavioral change, for better or for worse. 

The importance of reputable information and misinformation was clearly evident
in the recent and ongoing Covid-19 pandemic. While the vaccines against the
Covid-19 virus prevent severe illness, hospitalization, and death, only 60\% of
Arizonans over 12 years old are fully vaccinated against the virus. We know that
people exposed to misinformation are much less likely to receive vaccinations
against the virus \citep{loombaMeasuringImpactCOVID192021}. It is not an
exaggeration to claim that misinformation has been deadly for Arizona
\citep{pathakInfodemicsCOVID19Role2020, greene_murphy21}.

This dissertation on information search and norm development provides clear and
timely social implications for Arizona and the United States. This dissertation
uncovers how norms are created during periods of norm uncertainty and builds the
foundation to theorize and test interventions that promote positive norms in
times of disasters and uncertainty. Specifically, I investigate how local
diffusion leads to social norms through information search which will have
strong implications for the prevention of global polarization and the spread of
misinformation.

In academia and policy, the main focus in research on information has always
been the information that is sent to consumers, seeing people as passive
receivers of information. My dissertation instead focuses on individuals as
active agents in their search for information which helps me uncover the
information individuals are exposed to. By looking at how individuals search, I
aim to identify pain points and focus areas for future interventions in the
misinformation process.

The information that spread to communities in Arizona led to the health
behaviors exhibited today by Arizonans. While the information spread in various
ways, like through diffusion and polarization, it contributed to our newly
established norms in the Spring of 2020 and since. By uncovering how norms were
created during this time of norm uncertainty, my research will contribute to the
recommended interventions to promote effective health-related behavioral norms
before politicization and misinformation occurs. This will create a healthier,
happier Arizona.

\section{Theoretical Underpinnings}

\subsection{Information Flow}
% TODO what happened before this point? 
A critical turning point in the study of the flow information was put forth by
renowned Sociologist Paul Lazarsfeld \citeyearpar{lazarsfeldPeopleChoice1944}.
Lazarsfeld and others theorized that the media did not directly influence the
masses, but that a "Two-Step Flow" was occurring where "opinion leaders"
interpreted messages from mass media and then, in turn, promoted their opinions
to the masses \citep{katzPersonalInfluencePart1955}. The opinion leaders
interpret, explain, and diffuse contents to "opinion followers". From a networks
perspective, "people rarely act on mass-media information unless it is also
transmitted through personal ties" \citep[p. 1374]{granovetterStrengthWeakTies1973}. 
Lazarsfeld \citeyearpar{lazarsfeldPeopleChoice1944}'s first study focused on
politics but follow up studies from \citep{katzPersonalInfluencePart1955} showed
that this model matched information flows beyond the political realm. In
contrast to the one-step flow, where media directly influences the masses,
Lazarsfeld's theory remains in prominence today. While many scholars assumed
that modern micro-targeted marketing strategies would provide evidence for a
one-step flow perspective \citep{bennettOneStepFlowCommunication2006}, many
scholars find that "opinion leaders" (often influencers and celebrities) still
create a two-step flow of information transmission \citep{choi15,
hilbertOneStepTwo2017}.

The spread of information has been thoroughly studied from the prospective
of social networks. Individuals engage with each other and their distributive
ties to create community contexts where norms, beliefs, and values circulate.
Information diffuses through communities and social networks
\citep{fowler2010cooperative, bond_etal12, klarEffectNetworkStructure2017}.
However, "different things spread in different ways and to different extents"
\citep[p. 563]{christakisSocialContagionTheory2013} and information can spread
differently than a virus or behavior. Important studies in the spread of
information through networks include the study of adoption of innovations by
physicians \citep{colemanDiffusionInnovationPhysicians1957}, how new inventions
are adopted from early adopters to laggards
\citep{rogersDiffusionInnovations1962}, how weak ties help bind networks and
diffuse information \citep{granovetterStrengthWeakTies1973}, how social ties
spread information about job opportunities
\citep{granovetterGettingJobStudy1995, montgomeryJobSearchNetwork1992}, how
information spreads best when weak and strong ties are balanced (embeddedness)
\citep{uzziSocialStructureCompetition1997}, and more recent innovations in the
stickiness of complex contagion \citep{centolaComplexContagionsWeakness2007}.

\subsection{Information Search}
I only discussed the way that information is passively
consumed through push efforts by opinion leaders and social networks. However,
sociological literature is really sparse on the opposite behavior: active
directed searching by individuals to obtain information.
\citep{pejtersenDesignComputeraidedUsersystem1984}, a scholar of library and
information science, theorized that there are 5 strategies for searching for
information. The most common strategy is browsing, where people follow leads
based on associations without planning ahead. Another strategy is analytical,
which includes an explicit consideration of all facets of the question to guide
a search. The empirical method guides the search based on tactics that were
successful in past research. The known site strategy is to go to the direct
source of the information if known. And finally, the similarity method is to
find information based on another similar question that already has an answer.
These 5 strategies vary in their demands for prior knowledge, cognitive
processing, memory, and time spent. While this theory is aimed at finding
information in a library setting, scholars have extended the theory to other
fields and validated the framework \citep{fidelHumanInformationInteraction2012};
the frame is a useful beginning point for the theories of information search
through network activation or through computer-mediated communication.

\subsection{Information Search in The Digital Age}

The introduction of the internet and digital technologies introduced an epochal
disruption in the communication and informational-search mediums that people had
available to them. Computer-mediated communication was the largest communication
revolution in a century, changing the speed and availability of communication
between individuals by electronic mail, instant messaging, two-way interactive
video calls, discussion forums, blogs, and social networking
\citep{rainie2012networked}. Computer-mediated communication encompasses any
sort of communication through computers connected to the internet; the umbrella
term involves both synchronous and asynchronous communication as well as
one-to-one, one-to-many, or many-to-many exchanges. Social media, on the other
hand, are more specific, and are "Internet-based, disentrained, and persistent
channels of mass personal communication facilitating perceptions of interactions
among users, deriving value primarily from user-generated content.” \citep[][p.
50]{carr2015social}. While an email between colleagues is computer-mediated
communication, it is not social media as there is only one interaction between
users and there is no value derived from the user-generated content outside the
pair \citep{bayer_etal20}. Even more restrictive still is the social networking
site, which consists of a profile, a network, and a stream or a feed
\citep{boyd2007social, ellison2013sociality,},  often also involving a messaging
component  \citep{bayer_etal20}. Social networking sites are also interesting
because they often uniquely combine one-to-one communication within the context
of one-to-many communication, like resharing a television news story to your
facebook feed and chatting with others about the content in the comments
section. But what do these classifications mean for the information-seeking
behaviors of individuals in the 21st century? Not only is it quicker to perform
network activation and seek support, but people are "permanently online,
permanently connected" making activation easy \citep{vorderer2017permanently}.
The constant connectedness facilitates communication, but also presents people
with overchoice, or the difficulty in making a decision because simply every
network connection you may want to activate is permanently connected and easy to
reach. In other words, constant connection may be debilitating to many who are
seeking information. \citet{smallSomeoneTalk2017} provides many examples of this
when he demonstrates that more often than not, graduate students who needed
support simply asked "miscellaneous classmates encountered down the hall" (p.
176) rather than their trusted, long-term confidants. It seems that network
activation may rely on more than just trust; namely: convenience.

The digital age did not only revolutionize communication. It also
revoluitionized information search \citep{ramirez2002information}. Just as prior
to the digital age you could open an encyclopedia or peruse the library to find
the information you seek, much of the computer-mediated information search
surely occurs outside of social network activation contexts. While searching in
an encyclopedia is far less common today, internet search engines have
revolutionizes the speed and ease at which we can find information. Early web
search engines allowed information seekers to browse web directories and early
engines like Yahoo! and Altavista found success using keyword-based search,
though the early algorithms were limited in what they could find and relied
heavily on the wording of the queries. Google, the major web search engine used
today, introduced the disrputive PageRank algorithm in the early 2000s, securing
its place as the information hegemon of the 21st century
\citep{brin1998anatomy}. Not only has computer-mediated information search
revolutionized the ease of obtaining information, but it has also changed out
brains. \citet{sparrow2011google} show that the digital age has led to worse
memory and higher reliance on technology to keep track of information for us.
These search engines have also shaped our cultures by controlling, whether
through algorithm or intentionally, the information the engine provides. For
example, Google will not show results for various neo-Nazi websies in countries
where holocaust denial is illegal, like France and Germany. This is illustrative
of the control engines have over information and how this control may introduce
various political, economic, and social biases in the information they provide
\citep{segev2010google}. When 99Firms \citeyearpar{99firms22} compiled the data
from sources like NetMarketShare, Statista, DuckDuckGo, Yahoo Finance, Google,
and Pew Research they found that about about Around 93\% of all web traffic is
via a search engine and that 78.39\% of all internet search use Google. Because
computer-mediated information search is instantaneous and only requires mobile
data or wifi, there is virtually no cost to seeking information on one of these
mediums.

\subsection{Theory of Uncertainty Management}

Some communication theorists ask why an answer to a question is sought in the
first place. The Theory of Uncertainty Management
\citep{brashersCommunicationUncertaintyManagement2001} professes that people
search for information when their uncertainty around the subject leads to
anxiety or other cognitive harms. The Theory of Motivated Information Management
\citep{afifiSeekingInformationSexual2006, afifiTheoryMotivatedInformation2004}
extends the prior by adding that uncertainty itself is not the catalyst for
information-search; rather, it is driven by a discrepancy between the current
level of uncertainty on a subject and desired level of uncertainty.

\subsection{Social Support}

One way to find information is to activate network ties to find out information
through a form of social support. Social support, while previously used
interchangeably with the term social networks and social integration
\citep{houseStructuresProcessesSocial1988}, are the emotional, informational,
and instrumental assistance functions performed between social ties and have
strong and measurable association with health outcomes
\citep{houseMeasuresConceptsSocial1985, thoitsMechanismsLinkingSocial2011}.
Informational support is the process of seeking "help in defining,
understanding, and coping with problematic events and include education, advice,
or referral to another source of support" \citep[p. 640]{winemiller_etal93}.
Brashers, a health communications researcher, defines informational support
slightly differently, focusing on the exchange of information that "facilitates
coping with life stresses... that may be exchanged among members of a support
network" \citeyearpar[p. 260]{brashersInformationSeekingAvoiding2002}. Both of
these definitions provide important lenses for my question of how
computer-mediated or interpersonal information-seeking strategies vary. 
Some research has revealed that Facebook users who utilize the platform to 
find information and to maintain relationships have high levels of bridging
social capital \citep{liu2016meta}. However, this leaves the question 
of who activates their social networks for information using social networking 
sites like Facebook over activating these networks in face-to-face interactions. 


\subsection{Uses and Gratification Theory}

From studies of mass communication, uses and gratifications theory (UGT)
\citep{blumlerUsesMassCommunications1974, tanMassCommunicationTheories1985} may
lend itself to theorizing why information-seeking strategies vary among different
platforms and activation types. UGT posits that users are not passive
consumers of media and that people have an active role in choosing different
sources of media based on their satisfaction of specific needs on an individual
basis. UGT is based on Maslow's \citeyearpar{maslowTheoryHumanMotivation1943}
hierarchy of needs and is compatible with Lazarsfeld and Katz's theory of
two-step flow because people can choose their media and the opinion leaders they
follow. Modern-day theorists have extended UGT theory and classified the uses
and gratifications of the internet and of social media.
\citet{staffordDeterminingUsesGratifications2004} theorize that the internet
provides gratification through useful content that meets expectations,
gratification from purposeful navigating or random browsing as a process, and
social gratification from forming and deepening social ties.
\citet{leungGenerationalDifferencesContent2013} theorizes that social media is
gratifying for users because it allows for venting of negative feelings,
provides recognition, provides entertainment, promotes social affection, and
fulfills cognitive needs. 

There has been some research into how different platforms allow for different user 
experiences or "affordances" \citep{boyd2010social}. For example, \citet{bossetta18} first show
that all social media platforms must provide "tools that are easy to use and 
functional to the demands of varying using demographics" (p. 473). They also
explain how each digital architecture enables, contstrains and shapes how users
behave on their platform and in virtual space, leading to many different experiences 
along the dimensions of searchability, connectivity, and privacy. 

Adapting UGT to my own purposes, I theorize that
informational support will be activated from network ties when the informational
need is related to forming and deepening social ties
\citep{baumeister,grieve2013face} but search will be conducted outside of network
activation when information is needed but an individual does not have an additional cognitive
need to fill through social interaction. Alternative goals such as identity
management or relational maintenance
\citep{brashersInformationSeekingAvoiding2002} need to be balanced and may also
influence where search is conducted.

\subsection{Social Norms}
I showed the levels of uncertainty may direct how and when people
search for information. However, in situations of rapidly changing norms,
uncertainty is prevalent. This dissertation will also focus on developing
new social norms in situations of uncertainty about how to behave. 
Minimal research exists in quickly emerging
norms in times of emergency, but shows that people likely to rely on the
behavior of others' to set their own norms for behavior \citep{alvarez2018,
horneNormsIntegratedFramework2020}. This project will investigate how search
contributes to the development of said spontaneous norms.

Social norms form the building blocks of social organization and have been a
focus of sociology since the beginning. In a recent Annual Review of Sociology
piece focused on social norms, Horne and Mollborn define norms as "group-level
evaluations of behavior ... [or] when people have expectations about how others
evaluate behaviors" \citeyearpar[p. 468-69]{horneNormsIntegratedFramework2020}.
One of the principal architects of sociology, \'{E}mile Durkheim, saw that
society exerted powerful forces on individuals and classified people's norms,
beliefs, and values as parts of collective consciousness that provide for
societal integration \citeyearpar{durkheimSuicide1897, durkheimDivisionLaborSociety1933}.
For Durkheim, norms were therefore the glue that held society together in cohesion.

Max Weber, another founding architect of sociology saw social norms differently.
Weber distinguished social behavior from social action, which was an action a
person takes through subjective understanding and interpretation of actions of
others \citeyearpar{weber1978economy}. Weber therefore saw social norms through
the lens of the subjective meanings behind every action. Actions that were not
considered by an individual were not the focus of Weber. New spontaneous norms
that are emerging in a society under upheaval, like the health behaviors that
emerged during the Covid-19 pandemic, emerged with new subjective meanings and
cognitive associations that held consequences for the individuals who adopted
the social actions.

The causes and logics of social norms are debated. The consequentialist
argument for norms focuses merely on the consequence a behavior has on
group members and that norms will be promoted if the behavior benefits
others and denounced if it negatively affects others \citep{ullmannmargalitEmergenceNorms1977}.
 While this theory has been tested and performs well in the lab,
researchers have raised questions regarding its ecological validity
\citep{horneNormsIntegratedFramework2020}. One of the main critiques of the
consequentialist argument is that it relies on the same interpretation of the perceived action
by all parties of what is harmful and beneficial. The Relational argument 
for norms is based on the value people place on their relationships and
the assumption that people will behave in ways that will garner them
positive social reactions. This argument relies on individuals assuming
what their peers will approve and disapprove of. In turn, norms are
developed when individuals make inferences from their social worlds
\citep{fryeCulturalMeaningsAggregation2017}.

Importantly, people look to others performing an action and often
interpret that performance as an indication that an action is approved,
paying particular attention to high-status individuals and institutions
\citep{robalinoPeerEffectsAdolescent2018, tankardEffectSupremeCourt2017}.

\subsection{Social Networks}

Norms are formed inside of social networks; "Networks operate in a larger cultural
context that can facilitate or inhibit acceptance of general cultural norms and
beliefs" \citep[][p. 44]{pescosolidoDurkheimSuicideReligion1989}. 
Because of this, people tend to resemble
their network as a form of network autocorrelation and homophily
\citep{dellapostaWhyLiberalsDrink2015}. While "birds of a feather flock
together" in homophily \citep{mcphersonBirdsFeatherHomophily2001}, attitudes and
behaviors are shaped by peers, creating filter bubbles of social influence
\citep{dellapostaWhyLiberalsDrink2015}.

\citet{goldbergSocialContagionAssociative2018} argue that the state of the
literature on diffusion and social contagion, discussed above, is erroneous.
They propose an important linkage that had been missed in previous research of
the diffusion of behavior (or, as I interpret it, the creation of new norms).
They argue that what actually diffuses during social contagion are the
perceptions about which beliefs or behaviors are compatible with one another,
what they call "associative diffusion". 
This relates back to the theory of relational norms, where individuals assume
what their peers will approve and disapprove of. In order for these assumptions 
to take place, they must have an understanding of which behaviors and beliefs
are possessed by whom in their network and how new associations may influence
or disrupt the expected actions. In times of behavioral uncertainty, I
theorize that individuals strategically look towards other high-status
individuals, institutions, and perform regimented searches to establish for
themselves the behavior that is compatible with other cognitive biases they
hold.

\section{Three Empirical Articles}

To investigate information-seeking behaviors, online search tools, and how
new norms in health behaviors form, I first provide an outline of the 
information push and pull process in figure \ref{fig:concept-map}. After 
being exposed to new information, an actor can choose to further investigate
the information online or offline and, in turn, further share the 
information they were exposed to. 

\begin{figure}
{\centering \includegraphics[width=\linewidth]{figs/ch1/Dissertation Concept Map - Color.pdf}}
\caption{Theoretical Framework for the Dissertation}\label{fig:concept-map}
\end{figure}

This dissertation uses three articles to investigate different parts of
the aforementioned theoretical framework (figure \ref{fig:concept-map}).
The dissertation is divided into three parts, with Article 1 focusing 
on online information search, Article 2 focusing on online and offline
information search, and Article 3 investigating the entire information
push and pull process in the context of establishing new norms. 


\hyperlink{paper-1}{\subsection{Article 1: Digital Trace Data as Indicators of the Social World: Validating Google Trends for use in Scientific Research}}

Computational social science and other disciplines are quick to adopt new
sources of digital trace data for use in academic research. In this article, I
examine the validity of one of these new sources, Google Search Trends. I
validate how Google Trends can be used to demonstrate three different cases,
namely attitudes, disease prevalence and political preferences using eight %TODO double check number
different validated data sources. This paper asks, "How can we operationalize
Google Search Trends as a valid indicator for uses in social science research?"
I use Pearson and Repeated Measures Correlations between the Google Trends and
the validated indicators as well as multiple linear regression (for
cross-sectional datasets) and random intercept hierarchical linear models (for
longitudinal datasets) as additional tests of the data.

I fail to find correlation among any of the Google Trends tested and their
validated indicators. While some Google Trends tested were significantly
associated with the outcome in the regression models, effects tended to be small
and the total model interpretability remained low, even when controlling for
demographic variables. This article shows that there is no external validity of
Google Trends for these uses and social scientists will find no replacement for
high quality survey data with Google Trends.

\hyperlink{paper-2}{\subsection{Article 2: Information-Seeking Behaviors for Information on Covid-19 Vaccinations}}

Knowing that Google trends data only encompasses a small portion of the
information-seeking done by modern humans, it is important to
investigate what leads individuals to search for a question online using
a search engine (contributing to macro-level trends data) versus more
traditional means of information gathering, such as network activation
(asking someone in their social network for informational support). 
Previous research has greatly failed to distinguish between the activation of
information seeking behaviors online and offline. Using theories of social
support and uses and gratifications theory, I investigate the factors associated
with each vehicle type when on information seeking vehicle: personal connection,
doctor, social networking site, online forum, and online search engine. I use
original survey data of 948 Americans to investigate their experiences seeking out
information about the Covid-19 vaccines. 

This paper aims to address two main questions:
\begin{itemize}
  \item How do computer-mediated or interpersonal information-seeking strategies vary
across populations?
  \item How does the information search vehicle utilized affect Covid-19 Vaccination uptake? 
\end{itemize}

In this article, I find little evidence that online search is more utilized than
seeking social support from personal network connections or health professionals
as I hypothesize based on uses and gratifications theory. I find evidence that
the categorizations of vehicles queried in this survey are indeed conceptually
different from each other and that the utilization of different vehicles varies
by demography and information exposure points. Finally, I find that different
exposure points and information search vehicles hold real world consequences
through their associations with Covid-19 vaccination rates and intentions, as
information from a doctor increased the Covid-19 vaccination uptake while
receiving information from a Social Networking Site like Facebook or Twitter was
associated with lower odds of vaccination.

\hyperlink{paper-3}{\subsection{Article 3: Social Norms under Uncertain Times: A dynamic study of Stay-At-Home and Vaccination Rates During the Covid-19 Pandemic}}

My final article takes a deeper dive into the formation of social norms
governing health behaviors in cases of extreme uncertainty using the cases of
both stay-at-home rates and vaccination rates as responses to public health recommendations to mitigate the Covid-19 pandemic. Using theories of associative diffusion and the integrated
theoretical framework of norms, I test models of behavioral adaption to public health
recommendations as well as patterns of complex social contagion using linear
mixed effects models. 

These models show that complex contagion is a valid 
framework for the social contagion of new norms during Covid-19. Importantly, I find
important moderating effect of signal discordance, the contextual diversity of
signals received by an ego. This paper shows that the contagion process is not
fully understood without looking at the context of each exposure point within the
range of exposures one experiences. 


\section{Contributions and Future Directions}
\begin{center}
    \texttt{UNDERGOING CHANGES}
\end{center}

% TODO make Broader Impact more relevant to actual findings
% Original

This research will first and foremost impact sociology. While the spread
of information is investigated in the discipline (diffusion, mass
communication, social influence), it is almost always from a structural
perspective which disregards individual agency and choice in the
acquisition of new information. This perspective aligns with Uses and
Gratifications Theory, discussed in Section B: Background, and sees
people as active agents in their search for information instead of just
passive receivers of information signals.

Taking this agentic perspective is critical in the study of information
diffusion when combined with cognitive perspectives like that of
Goldberg and Stein (2018). How people search for information determines
the information that they find; the information they uncover is filtered
through cognitive biases and predispositions to how they interpret and
act upon the newfound information. Searching for informational support
through social network ties theoretically will uncover potentially
different information than that would be uncovered through online web
search and will change how people are organized and how people behave.

Additionally, this dissertation will investigate how social norms are
formed in situations of uncertainty, an area that is less developed in
sociology. While sociologists have studied social norms since the
formation of the discipline, focus on the quick formation of social
norms has implications for social behavioral outcomes, cultural change,
organizational reactions to unrest and disasters, and many more areas
that influence humanity.

This dissertation will also have implications for the study of social
networks and social support. Not only is this work situated in the
sociological study of social support and social influence, but I utilize
rarely-used-in-sociology sources of big data to better understand these
mechanisms and test existing theories.

This dissertation also contributes to computational social science \&
critical big data studies. I base much of my dissertation around Google
Search Trends, which are relatively underutilized in the social sciences
compared to health sciences and business. Not only does this project
introduce Google Search Trends in new ways, but I perform explicit tests
of how this data should be used in the social sciences, providing the
groundwork for this exciting source of big data for future computational
social scientists. Because my research takes a critical big data studies
approach, I take into account the potential pitfalls with any source of
big data (McFarland and McFarland 2015) and attempt to explicitly answer
questions posed by this field, such as "what are the behavioral
processes that lead to macro-level outcomes" proposed by \citet{breigerScaling2015}.

I also make important contributions to social epidemiology through this
research. Each of the three papers in this dissertation use health case
studies for questions of sociological interest. Because of this, I will
contribute to social epidemiological questions on vaccine hesitancy,
health communication, and diffusion of high-risk health behaviors. While
all of these areas are important to social epidemiology, there are
potential important health implications for these contributions on
outside of the academy.

Outside of the academic research impacts that this dissertation makes,
there are clear and timely social implications that will stem from this
work. First, by uncovering how norms are created during norm
uncertainty, I may be able to establish recommendations for
interventions to be applied during disasters and other times of unrest
to promote effective positive norms. Second, my research may have
implications for how to optimize gratifications in search processes to
promote healthy social relationships and improved online algorithm
deployment.

This work also relates to polarization and misinformation, two major
social issues of our time. \citet{axelrodDisseminationCultureModel1997} shows that network structure,
autocorrelation, and diffusion create local convergence of attitudes and
culture which then leads to global polarization \citep{dellapostaWhyLiberalsDrink2015}. In investigating how local diffusion leads to social norms
through information search, some of my findings may have strong
implications and interventions for the prevention of global
polarization. Relatedly, understanding where and how people search for
information has implications for the information they are exposed to and
the norms they develop. This project will attempt to relate to where
misinformation is uncovered and how to intervene in its absorption.


\section{Conclusions}
\begin{center}
    \texttt{UNDERGOING CHANGES}
\end{center}


what do these papers come TOGERTHER to tell us about information search?
