\hypertarget{paper-3}{\chapter{Social Norms under Uncertain Times: A dynamic study of Stay-At-Home and Vaccination Rates During the COVID-19 Pandemic}\label{paper-3}}

\hypertarget{abstract-1}{\section{Abstract}\label{abstract-1}}

My final article takes a deeper dive into the formation of social norms
governing health behaviors in cases of extreme uncertainty using the cases of
both stay-at-home rates and vaccination rates as responses to public health
recommendations to mitigate the COVID-19 pandemic. Using theories like
associative diffusion and the integrated theoretical framework of norms, I test
models of behavioral adaption to public health recommendations and patterns of
complex contagion (the need for repeated exposures to something novel for it
to diffuse) using linear mixed effects models. These models show that complex
contagion is a valid framework for the social contagion of new norms during
COVID-19. Importantly, I find a novel moderating effect of signal discordance,
the contextual diversity of signals received by an ego. If there is diversity in
the information received by an ego, contagion is less likely to occur. This
paper shows that the contagion process cannot be fully understood without
looking at the context of each exposure to a contagion within the range of
contagions one experiences.

Keywords: Social Networks, Complex Contagion, Social Norms, Health Behavior, Diffusion

\hypertarget{background}{\section{Background}\label{background}}

After spreading around the world in a matter of months, the coronavirus
(COVID-19) became a leading cause of death in the United States. Although the
Centers for Disease Control and Prevention (CDC)
\citeyearpar{centersfordiseasecontrolandpreventionHowProtectYourself2020}
proposed several potential mitigation strategies, the method of mitigation that
received the most national attention is the call to stay-at-home and social
distance for non-essential workers. CDC officials and front line health care
professionals advise that the best way to prevent exposure to the virus is to
stay-at-home and avoid close contact with people.

A second strategy pushed by the CDC, once available, was to receive vaccination
to prevent the effects of the COVID-19 virus if infected. Vaccinations like
Pfizer-BioNTech (Comirnaty, tozinameran, BNT162b2), and Janssen/Johnson \&
Johnson, Moderna (mRNA-1273) started receiving approval for emergency use
authorization by the United States Food and Drug Administration (FDA) in
December 2020.

Although these public health interventions were pushed by policy makers
throughout 2020 and 2021, policy makers often struggle to promote public
adoption of policies that rely on the publics' willingness to adjust their
current `risky' behaviors to less-risk behaviors based on scientific facts. The
lack of adaption to newly recommended behaviors is partially due to the
variation in how individuals came to understand the threats of the disease
\citep{akpanAssociationWhatPeople2021, bailey_etal20}, but also due to habitus:
the ingrained habits, skills and dispositions of individuals and groups
\citep{bourdieu77}. Habitus shapes how individuals perceive different social
interactions and norms and shape how they adopt different policies, leading to
different social outcomes and a divergence of attitudes
\citep{scottarthur_etal21, williams95, madeira_etal18}.

There is little research on stay-at-home rates, especially in cases of pandemics
or disasters. Though research on the causes of population mobility did expand
after the onset of the COVID-19 pandemic
\citep{bargainTrustCompliancePublic2020, bourassaStatelevelStayathomeOrders2020,
bourassaSocialDistancingHealth2020, grossmanPoliticalPartisanshipInfluences2020,
haggerPredictingSocialDistancing2020, hillBloodChristCompels2020,
hillNastiestQuestion, hillLoveThyAged2021,huynhDoesCultureMatter2020}, most
research focused on unchanging cultural determinants of the reduction in
stay-at-home rates. For instance, \citet{gibbons_etal21} finds that social
capital had spatially inconsistent effects on stay-at-home rates. Instead of
focusing on cultural relationships with stay-at-home rates, this paper will
instead focus on the dynamic nature of population mobility across population
networks.

There is more research on vaccine uptake
\citep{schmidBarriersInfluenzaVaccination2017} because of increased hesitancy
and anti-vaccination movements over previous decades
\citep{baumgaertnerInfluencePoliticalIdeology2018,
hornseyDonaldTrumpVaccination2020, johnsonOnlineCompetitionPro2020,
whiteheadHowCultureWars2020}. While some studies use social network analysis to
study the discussions for an against vaccinations online
\citep{milaniVisualVaccineDebate2020}, vaccination rates are a great way to
investigate the establishment of new social norms. First, vaccination uptake is
measurable with stable information tracking infrastructure and vaccination rates
are varied across locations and time. Secondly, because vaccines have become
politicized \citep{mottaRepublicansNotDemocrats2021,
mottaIdentifyingPrevalenceCorrelates2021}, the social contagion effects can be
studied.

This article researches the formation of social norms governing health behaviors
in cases of extreme uncertainty during the COVID-19 pandemic. Using theories of
complex contagion, associative diffusion and the integrated theoretical
framework of norms, I test how these health behaviors are outcomes of contagion
and discordance of signals, as well as other factors affecting associations like
religious and political conservatism, attention to television media sources, and
online norms as measured by information-search behaviors.

Before outlining my
central hypotheses, I offer an overview of the current state of the social
contagion literature. After that, I will outline how I compiled this unique
longitudinal data set from Google, the CDC, Facebook and other sources and how I
define and calculate signal and discordance, two key independent variables for
this paper. To model stay-at-home and vaccination rates, I use linear mixed
effects models with random effects as a longitudinal model of contagion. After
interpreting the results of the models, I discuss the implications of these
findings for the sociological understanding of social contagion and the
formation of social norms.

\hypertarget{the-social-contagion-model}{\subsection{The Social Contagion Model}\label{the-social-contagion-model}}

Individuals engage with each other and their distributed ties to create
community contexts where norms, beliefs, and values circulate. These clusters of
interaction are called social networks, and if ``each person continues to
interact primarily with others nearby in space, the forces of conformity will be
strongest locally, leading to the emergence of clusters of people sharing
similar behavior'' \citep{kitts_shi18}. This community interaction ultimately
leads to converged communities of belief structures with variations in how much
they diverge from the norm \citep{cullumCulturalEvolutionInterpersonal2007,
lataneExperimentalEvidenceDynamic1996, okadaStructureCulturalRejection2017}.

``Culture'' diffuses through communities and social networks. Information and
opinions spread \citep{bond_etal12,
fowler2010cooperative,klarEffectNetworkStructure2017}, behaviors are adopted
\citep{aralExerciseContagionGlobal2017, centolaSpreadBehaviorOnline2010,
centolaExperimentalStudyHomophily2011,christakis2008collective,rosenquist2010spread},
and there are patterns of health contagion 
\citep{cacioppo2009alone,christakisSpreadObesityLarge2007}. However, ``different
things spread in different ways and to different extents'' \citep[p.
563]{christakisSocialContagionTheory2013} and when modeling diffusion and
contagion, we must be very specific about our scope conditions as they are
relevant to our theory and not to cross theories to infer connections where they
may not exist \citep{kitts_quintane20}.

Most of the diffusion literature does not focus on establishment of new norms
but the adoption of culture and specific deviant behaviors \citep[see][for an
exception]{centolaSpontaneousEmergenceConventions2015}.
\citet{dellapostaWhyLiberalsDrink2015} outline how the spread of culture and
behavior is tied to network autocorrelation, or ``the tendency for people to
resemble their network neighbors.'' They show that the distance between two
agents in sociocultural space can determine the likelihood of the adoption of a
new behavior. Like \citet{axelrodDisseminationCultureModel1997}, this outlines
how the local convergence of close network actors becomes amplified and can lead
to global polarization between groups.

Outdated models found in early public health research claim that information
about the risks of behaviors will lead to changes in behaviors to mitigate those
said risks \citep[e.g.][]{flay_etal80}. However, while this model can be valid
in specific cases, more research has shown how the risks themselves do little to
motivate behavioral change \citep{witte_allen00,wolburg06}. Often, an appeal to
fear is found to be a major driver of the adoption of information campaigns and
some research has shown that ``social network exposure to COVID-19 cases shapes
individuals'' beliefs and behaviors concerning the coronavirus''
\citep{bailey_etal20}. Because of this, it is logical that higher local
incidences of infection would inspire fear and increase adherence to public
health measures according to this older model.

\begin{hyp2}[H\ref{hyp:infection}] \label{hyp:infection}
 Relatively higher local rates of infection will lead to increased time spent in residence and increase vaccination uptake
\end{hyp2}


\hypertarget{complex-contagion-and-discordance}{\subsection{Complex Contagion and Discordance}\label{complex-contagion-and-discordance}}

\citet{centolaComplexContagionsWeakness2007} theorize that simple contagions are
not enough to spread behavioral change. Simple contagions are those in which
only one point of contact is needed to receive contagion, like with infectious
disease or to learn simple bits of information. Centola and Macy
\citeyearpar{centolaComplexContagionsWeakness2007}'s large contribution was the
theorizing of complex contagions, those that require ``independent affirmation
or reinforcement from multiple sources'' (p. 703) and is not based on the number
of exposures but the number of sources of exposure. In the case of behavioral
contagion, this means that the behavior must be reinforced through witnessing
multiple alters perform this behavior before contagion can take effect.

In the case of stay-at-home rates or vaccination rates, this theoretically means
that a person exposed to many sources of the same signal (high, medium, low
rates) will be more likely to adopt the behavior based on the reinforcement from
the multiple sources of exposure. As norms are inherently social, I theorize
that we can see complex contagion happening in real time with the following
hypothesis:

\begin{hyp2}[H\ref{hyp:direction}] \label{hyp:direction}
Increased average time spent in residence (signal direction) from alters will have a positive effect on stay-at-home rates the ego-county; increased vaccine uptake by alters will have a positive effect on vaccine uptake for the ego-county
\end{hyp2}
% TODO I think I ``get'' this, but could you define ``ego county''? Also, why not just look at the effect of alter's behavior on ego's behavior (why does the county come into the framework)? Do you mean {the subset of ego's alters who live in the same county as ego}?

To make the contagion more complex, different sources of exposure
(county-alters) adhere to CDC recommendations to stay-at-home at differing
rates. Whereas one county-alter may be greatly increasing its time in residence,
another may have made little change. When the majority of alters is in
concordance with each other, the signal to the ego is reinforced and more
impactful on the ego. When these signals are mixed with high variance from
different sources, agreement is low and makes the behavioral change less likely.
For this paper, I theorize that a new concept of `discordance' must be
considered as impacting complex contagion. Discordance is the variance of
signals received by an alter; high discordance prevents reinforcement while low
discordance (concordance) enables complex contagion. Instead of adopting a
`majority' rules attitude, this means that the more discordance perceived by an
ego, the less likely the alters will have any effect on the ego. I theorize that
the behavior of a county-alter will be correlated with the behavior
of another county-ego if the county-ego receives highly discordant signals, as
seen in Figure \ref{fig:dag}.

\begin{hyp2}[H\ref{hyp:discordance}] \label{hyp:discordance}
The effect of signal direction on time spent in residence and vaccine uptake will be moderated by diversity in signals (discordance)
\end{hyp2}

\begin{figure}
{\centering \includegraphics[width=0.8\linewidth]{figs/paper3/dag}}
\caption{Elaboratory Theoretical Model of Health Behavior Norms}\label{fig:dag}
\end{figure}

\hypertarget{associative-diffusion}{\subsection{Associative Diffusion}\label{associative-diffusion}}

While much of the social contagion literature, like the theories above, focuses
on structural boundaries and homophily as causes of how diffusion occurs,
\citet{goldbergSocialContagionAssociative2018} propose a disrupting alternative
mechanism. They argue that what actually diffuses during social contagion are
the perceptions about which beliefs or behaviors are compatible with one
another, what they call ``associative diffusion.'' This argument that culture
does not spread like a virus but instead is dependent on how belief structures
are connected to each other is important to test because norms around health
behaviors became politicized issues during the COVID-19 pandemic. This means
that the stay-at-home or vaccination behavior themselves were not contagious,
but the cognitive association of precisely what social distancing or vaccination
uptake \emph{meant} were spreading between populations of individuals. Moreover,
as \citet{houghton20} outlines, diffusants, the things that are diffusing
through a population, are not independent of each other \citep{mason_etal07}.
When we take into account the interdependence between different beliefs that are
diffusing, such as COVID-19 is a hoax, social distancing saves lives, the
COVID-19 vaccine includes a microchip to control your thoughts, and lockdown is
against the rights guaranteed in the US constitution, we can imagine how these
various diffusants become associated together into structures of belief through
schemas of perception \citep{houghton20}. While this cognitive theory of
cultural variation is difficult to test, the theory it supplies provides a solid
framework for how behavioral norms formed during the pandemic.

Individuals look to norms to regulate behavior, avoid deviance, and to maintain
order \citep{horneNormsIntegratedFramework2020,
shepherdStructurePerceptionHow2017}. When they realize they don't have a
normative set of responses in their cultural toolkit to respond to an unfamiliar
situation they are presented with, individuals look to the other sources in
their ``community'' to mimic behavior, such as high-status individuals,
institutions, and members of their social network. I theorize that individuals
look towards their physical community, to their social network, popular media,
established norms that they may find online through search, and to the real
threat of infection (what would happen if I do nothing about this norm).
Following both the Integrated Theoretical Framework of Norms
\citep{horneNormsIntegratedFramework2020} and associative diffusion
\citep{dellapostaWhyLiberalsDrink2015, goldbergSocialContagionAssociative2018},
I theorize that individuals interpret signals from their ``community'' through
their cognitive biases and behavioral predispositions to determine their
formation of new behavioral norms.

There is some evidence I can draw upon to support the associative diffusion
framework. For instance, there is a central conflict between religious and
scientific ideologies which I theorize leads more religious counties to reject
the stay-at-home order and vaccines
\citep{evansReligionScienceEpistemological2008} because the associative
diffusion of these behaviors pits public health recommendations against
religious ideologies. Furthermore, as COVID-19 has become a politically
polarizing issue \citep{ternullo22}, conservative ideology and a general
mistrust of ``big government'' \citep{frank2007, gauchat2008} are likely to lead
to resistance to the government and scientific guidance.

\hypertarget{data-and-methods}{\section{Data and Methods}\label{data-and-methods}}

This article draws on two unique longitudinal datasets that I compiled for this
analysis from sources like Facebook, Google, CDC, and more. Because the
different datasets were on different scales (individual, zip-code, county,
state, designated market area) and time points (cross-sectional, longitudinal),
there was extensive data wrangling that had to be done to prepare these data for
statistical analysis. The steps taken to prepare each of the different model
features and where the data was sourced from is therefore covered in the
following sections before diving into modeling specifications.

\hypertarget{stay-at-home-rates}{\subsection{Stay-at-Home Rates}\label{stay-at-home-rates}}

\begin{table}[!ht]

\caption{\label{tab:google-desc-table}Descriptive Statistics for Stay-at-Home Models (county-level)}
\centering
\begin{tabular}{lrrrr}
\toprule
  & min & max & mean & sd\\
\midrule
Percent White & \num{0.14} & \num{0.98} & \num{0.81} & \num{0.15}\\
Percent College Graduates & \num{0.08} & \num{0.75} & \num{0.26} & \num{0.10}\\
Percent over 65 & \num{6.60} & \num{56.70} & \num{16.90} & \num{4.05}\\
Median Income & \num{28951.00} & \num{142299.00} & \num{59773.61} & \num{15264.50}\\
Monthly Unemployment Rate & \num{1.40} & \num{34.60} & \num{8.12} & \num{4.00}\\
Percent of GOP votes, 2016 & \num{0.08} & \num{0.90} & \num{0.57} & \num{0.15}\\
Percent Evangelical Christian & \num{0.00} & \num{0.67} & \num{0.20} & \num{0.14}\\
'Fox News' Trend & \num{0.00} & \num{204.08} & \num{29.59} & \num{16.94}\\
'Social Distancing' Trend & \num{0.00} & \num{500.00} & \num{20.64} & \num{25.36}\\
'Covid Conspiracy' Trend & \num{0.00} & \num{825.00} & \num{42.51} & \num{80.44}\\
Covid Case Rate & \num{0.00} & \num{343.71} & \num{21.17} & \num{27.44}\\
Week Number & \num{0.00} & \num{43.00} & \num{23.14} & \num{11.77}\\
Movement Signal & \num{-2.06} & \num{26.44} & \num{8.31} & \num{4.22}\\
Movement Discordance & \num{0.15} & \num{11.05} & \num{2.22} & \num{0.80}\\
\bottomrule
\multicolumn{5}{l}{\rule{0pt}{1em}Raw values presented in table. Values in models are normalized.}\\
\multicolumn{5}{l}{\rule{0pt}{1em}Notes: 1,375 counties, March 02 through December 28, 2020.}\\
\end{tabular}
\end{table}
%TODO define normalization, make sure its mentioned in the methods section

\begin{figure}
{\centering \includegraphics[width=0.8\linewidth]{figs/paper3/plot-google-1}}
\caption{Stay-at-Home Rates over Time}\label{fig:plot-google}
\end{figure}

My first dependent variable aims to operationalize behavioral norms as
stay-at-home rates with data from Google \citep{google2020}. While the Google
COVID-19 Community Mobility Reports include multiple measures of mobility based
on location and activity information, the change in time spent in residence most
closely aligns to an operationalization of an obedience to CDC and governor
orders, a new and emerging norm in 2020. The change in time spent in residence
variable represents percent change of time spent from Google Location History
within geographic areas that Google has designated as a residential area. These
data are aggregated to the county-level based on anonymized sets of data from
users who have turned on the Location History setting, which is off by default.
This means our sample is possibly skewed to people in the United States a) with
a mobile phone, b) with a Google account, and c) with knowledge of how to
synchronize their location history. Google has not made it clear whether there
are any weights in place to correct for these potential biases. Google
calculates the relative change in mobility in comparison to the median value of
movement in the area for the same corresponding day of the week, during the
5-week period Jan 3--Feb 6, 2020. As Google did not provide data on certain US
counties if they did not have sufficient statistically significant levels of
data, my final county sample is \emph{n} = 1375 (compared to the total
population of 3,107 US counties). Particular areas that are excluded in this
analysis include all counties in Alaska and DC, and over half of the counties in
North Dakota, South Dakota. A full list of excluded counties available upon
request. These data are longitudinal measures calculated by creating a moving
average of stay-at-home rates for each county (rolling window width = 7 days)
and then sampling every Monday in the sample for 44 measures from March 02, 2020
through December 28, 2020.

\hypertarget{covid-vaccination-uptake}{\subsection{COVID vaccination uptake}\label{covid-vaccination-uptake}}

\begin{table}[!h]

\caption{\label{tab:vacc-desc-table}Descriptive Statistics for Vaccination Models (county-level)}
\centering
\begin{tabular}[t]{lrrrr}
\toprule
  & min & max & mean & sd\\
\midrule
Percent White & \num{0.08} & \num{1.00} & \num{0.83} & \num{0.17}\\
Percent College Graduates & \num{0.03} & \num{0.78} & \num{0.22} & \num{0.10}\\
Percent over 65 & \num{3.20} & \num{56.70} & \num{18.86} & \num{4.50}\\
Median Income & \num{21504.00} & \num{142299.00} & \num{53522.45} & \num{14312.25}\\
Monthly Unemployment Rate & \num{0.90} & \num{22.00} & \num{4.96} & \num{1.95}\\
Percent of GOP votes, 2020 & \num{0.09} & \num{0.94} & \num{0.64} & \num{0.16}\\
Percent Evangelical Christian & \num{0.00} & \num{1.31} & \num{0.22} & \num{0.16}\\
'Fox News' Trend & \num{5.00} & \num{405.88} & \num{72.12} & \num{27.38}\\
'Covid-19 vaccine' Trend & \num{0.00} & \num{755.56} & \num{74.67} & \num{70.33}\\
'Covid Conspiracy' Trend & \num{0.00} & \num{2000.00} & \num{56.05} & \num{189.61}\\
Covid Case Rate & \num{0.00} & \num{998.58} & \num{22.79} & \num{27.23}\\
Week Number & \num{1.00} & \num{35.00} & \num{18.00} & \num{10.10}\\
Vaccination Signal & \num{0.00} & \num{62.84} & \num{19.62} & \num{14.84}\\
Vaccination Discordance & \num{0.00} & \num{22.40} & \num{5.26} & \num{3.62}\\
\bottomrule
\multicolumn{5}{l}{\rule{0pt}{1em}Raw values presented in table. Values in Models are normalized.}\\
\multicolumn{5}{l}{\rule{0pt}{1em}Notes: 2,819 counties, January 04 through August 30, 2021.}\\
\end{tabular}
\end{table}

\begin{figure}
{\centering \includegraphics[width=0.8\linewidth]{figs/paper3/plot-vacc-1}}
\caption{Vaccination Rates over Time}\label{fig:plot-vacc}
\end{figure}
% TODO add which counties rose to 100% vaccination 

The second dependent variable aims to operationalize vaccination uptake through
vaccination rate information in the United States from January 2021 through
August 30 2021 \citep{cdcCOVID19Vaccination2021}. Data represents all vaccine
partners including jurisdictional partner clinics, retail pharmacies, long-term
care facilities, dialysis centers, Federal Emergency Management Agency and
Health Resources and Services Administration partner sites, and federal entity
facilities. Vaccination data is available for all US counties with the exception
of parts of California and Massachusetts, Hawaii, and Texas. In Texas and
Hawaii, no county level information is available, and California does not report
the county of residence for vaccinations when the county of residence has a
population less than 20,000 people. Finally, Massachusetts does not provide
vaccination data for Barnstable, Dukes, and Nantucket counties because of their
small populations. Therefore, my final county sample for vaccination rates is
\emph{n} = 2819 (compared to the total population of 3,107 US counties). A list
of included FIPS are available upon request. Vaccination rates are scaled with a
rolling mean with a rolling window width of seven days to smooth out intra-week
noise. I sample every Monday between January 2021 through August 30, 2021 for 35
total observations for each county.

\hypertarget{independent-variables-1}{\subsection{Independent Variables}\label{independent-variables-1}}

\hypertarget{network-signal}{\subsubsection{Network Signal}\label{network-signal}}

I utilize the likelihood of a friendship connection between counties to create a
county-level social network weighted by the probability of a tie. Using this
network and the two independent variables above, I examine how the vaccination
and stay-at-home rates of peer counties is contagious to the ego-county. To do
this, I first take the weighted average of each ego's network signals using
equation \eqref{eq:networksignal} where \texttt{x} denotes the vaccination and
stay-at-home rates of each alter county and \texttt{w} represents the likelihood
of a friendship connection between the ego county and each of their alters,
lagged by one week. This `signal' of norms gives us insight into the co-evolving
contagion patterns of the new social norm being established. The likelihood of
friendship connection comes from the ``Social Connectedness Index''
\citep{Bailey2018, facebook20} which indexes the social links between
geographies by the likelihood of Facebook friends. It is an aggregated measure
of Facebook friendship connections between counties. It corresponds to ``the
(relative) probability that two arbitrary Facebook users across two geographies
are friends with each other.'' The data is available at various geographic
aggregation levels, such as U.S. zip codes or entire countries. However, data is
only available at one time point because Facebook has found that although there
are individual changes in friendship connections over time, the aggregate
statistical probabilities of friendship remain stable. This measure is based on
user-provided information and real-time location-service data gathered by
Facebook. Facebook data has been shown to be highly representative of the U.S.
population and Facebook friendship links largely represent real-world
friendships \citep{bailey_etal18, jones_etal13}. The data has been tested to
show how initial COVID-19 hot spots are related to subsequent virus spread to
non-hot-spots, even after controlling for population density and geographic
distance \citep{Kuchler2020}.

\begin{equation}
signal = \frac{\sum_{i=1}^nw_ix_i}{\sum^n_{i=1}w_i} \label{eq:networksignal}
\end{equation}

\hypertarget{signal-discordance}{\subsubsection{Signal Discordance}\label{signal-discordance}}

The second independent variable, \emph{Signal Discordance}, builds on the
network signal but looks specifically at the extent to which a given ego-network
receives diverse contagion signals. Based on Hypothesis \ref{hyp:discordance},
that when the majority of alters is in concordance with each other, the signal
to the ego is reinforced and more potent on the ego, a high discordance
coefficient is indicative of diverse signals which may prevent any clear
interpretation of a norm developing, whereas a low discordance coefficient would
indicate reinforced signaling. I use the formula to calculate weighted standard
deviations, (see equation \eqref{eq:signaldiscordance}) to provide a metric of a
diversity of signals. In this formula, \texttt{x} denotes the vaccination and
stay-at-home rates of each alter county and \texttt{w} represents the likelihood
of a friendship connection between the ego county and each of their alters,
lagged by one week. Furthermore, \(\bar{x}^*\) represents the weighted mean.

\begin{equation}
discordance = \sqrt{\frac{\sum_{i=1}^nw_i(x_i-\bar{x}^*)^2}{\sum^n_{i=1}w_i}} \label{eq:signaldiscordance}
\end{equation}

Figures \ref{fig:assortativity-vacc} and \ref{fig:discordancenetwork} provide a
visualization of both network signal and signal discordance for April 26, 2021
for 7 selected counties.\footnote{Figures in this paper were all created using
ggplot2 \citep{wickham_etal, wickham11}, patchwork \citep{pedersen20}, and
ggridges \citep{ggridges}} While a county like Lake County, Ohio has low
discordance meaning the signal of vaccination rate is reinforced, Navajo County,
Arizona receives very diverse signals from their county-alters, negating any
contagion effects. Figure \ref{fig:discordancenetwork} explicitly outline how
the weights and signals are calculated for an ego with only 5 alters.
% TODO Navajo County, AZ, is probably a good one to talk about. Suppose it's composed of Native and non-Native communities, which are not in much communication with each other. Each (Native and non-Native) might have connections to different sets of alter counties. In brief: There could be a lot of homogeneity among non-Natives and a lot of homogeneity among Natives, but both might be masked by taking county as the unit. Thoughts? the assumption of homogeneity within county  (my comment on p. 120) should be stated clearly.

\begin{figure}
{\centering \includegraphics[width=0.7\linewidth]{figs/paper3/assortativity-vacc-1}}
\caption{How Network Signal and Discordance are Calculated, Sampled Regions}\label{fig:assortativity-vacc}
\end{figure}

\begin{figure}
\begin{center}\includegraphics[width=0.5\linewidth,]{figs/paper3/discordancenetwork-1.pdf} 
  \begin{equation}
    w = \begin{bmatrix}0.0000015\\0.0000023\\0.0000527\\0.0000966\\0.0001119\end{bmatrix},  
    x = \begin{bmatrix}29.9\\20.4\\27.7\\24.6\\24.3\end{bmatrix} \longrightarrow
    \begin{matrix} \bar{x}*, signal = 25.083\\ s*, discordance = 1.423 \end{matrix} 
  \end{equation}
  \caption{How Network Signal and Discordance are Calculated, Mathematical Example}
  \label{fig:discordancenetwork}
  \end{center}
\end{figure}

\hypertarget{COVID-19-case-rates}{\subsubsection{COVID-19 Case Rates}\label{COVID-19-case-rates}}

To estimate the concept of `real threat of infection' for a fear-based model of
health behaviors, the models use county-level COVID-19 case rates with data
compiled by The New York Times \citeyearpar{covid_data}. It is widely
acknowledged that there are biases in this data due to inconsistencies and
availability in testing as well as different community propensity to test
\citep{gu22, cdc20a}. However, it is the best measure we have of actual case
rates. County data are scaled with a rolling mean with a rolling window width of
seven days to smooth out intra-week noise. Case Rate is measured as number of
cases per 100,000 population. Observations vary from a minimum of 0 to a maximum
of 1,565 across the two datasets.

\hypertarget{online-norms}{\subsubsection{Online Norms}\label{online-norms}}

To operationalize the search for online norms, I use Google search trends
\citep{googletrends}. In \hyperlink{paper-1}{Article 1}, I show
that Google Search Trends can only be used to indicate interest due to their low
criterion validity \citep{jungherr_etal17}. I use Google Trends over the study
period across individual designated media markets areas (DMAs), a nonoverlapping
aggregation of U.S. counties to 210 media markets based on similar population
clusters \citep{dma_key}. To investigate the rate of \emph{searching} for norms
online in both cases, I use the following search topics: `Social Distancing'
(2020, stay-at-home case only), `COVID-19 Vaccine' (2021, vaccination uptake
case only) and Covid Conspiracy (2020-2021, both cases). Search topics are a
more robust measurement than a single search term: topics are aggregations of
the rates of multiple, highly correlated search terms together into a cohesive
topic. For example, while `Beyoncé,' `Beyonce' and `beyonce knowles' are all
separate search terms, `Beyoncé Knowles' encompasses all of these into a single
search topic. While the data is originally on a scale of 0 to 100, with 100
being the maximum search popularity out of all DMAs, Google Search Trend are now
only available cross-sectionally (a single time period across a geography) or
time-series (a single geo-location across time). To remedy this and build a
longitudinal dataset of each search topic, I follow the method proposed in
\citet[p. 5]{park_etal}. This method involves building a dataset of unscaled
cross-sectional values, selecting a DMA to use to establish the rescaling ratio
(I use `Los Angeles CA'), and then finding the time-series values for the one
DMA. To find the rescaling ratio for each week in the time-series, you divide
the time-series value for each week by the cross-sectional value for each week,
resulting in a rescaling vector to be used for all weeks in the dataset across
geographies. To rescale each longitudinal value, multiply the respective week's
rescaling ratio by the cross-sectional value. Rescaled longitudinal data was
compared against time-series data for multiple test counties and was equivalent.
For a more in-depth explanation of this procedure, see \citet[p. 5]{park_etal}.

\hypertarget{pillars-of-convervatism}{\subsubsection{Pillars of Convervatism}\label{pillars-of-convervatism}}

Research has shown that stay-at-home rates and other pandemic health behaviors
are related to various `pillars of conservatism'
\citep{gonzalez_etal21,hillBloodChristCompels2020,hillLoveThyAged2021,
hillNastiestQuestion}. Namely, research shows that politically conservative
indicators, such as Republican political leadership, conservative Protestantism
and consumption of right-wing media are related to higher rates of movement and
lower rates of mask usage. Based on these findings, this article controls for
these factors in the following ways: to measure Republican political leadership,
I include percentage of votes for Donald Trump in the previous presidential
election; for the 2020 study, I use the 2016 results and for the 2021 study, I
use the 2020 results to infer proper time ordering. Second, to measure
conservative Protestantism, I employ the county's percentage of evangelical
Christians. These county-level data were collected through the 2010 U.S.
Religion Census: Religious Congregations and Membership Study
\citep{grammich_etal18}. Finally, right-wing media consumption is assessed using
Google Search Trends to capture ``Fox News'' searches over the study period
across individual designated media markets areas (DMAs). I use this measure to
indicate active interest in and attention toward Fox News. While Google Search Trends
have been validated for use in a range of research contexts and for use with
survey data, voting data, and ecological data
\citep{bailPrestigeProximityPrejudice2019, reyesUsingInternetSearch2018,
scheitleGoogleInsightsSearch2011, stephensdavidowitzCostRacialAnimus2014,
swearingenGoogleInsightsSenate2014}, I provide a caution for these uses in 
\hyperlink{paper-1}{Appendix A} and suggest these measures should only be used to measure attention. 


\hypertarget{demographics}{\subsubsection{Demographics}\label{demographics}}

These models also control for (1) percent of the population that is above 65
years old (those most at risk of hospitalization), (2) percent of the population
that identifies as white, (3) percent of the population that holds a college
degree, (4) median income, and (5) monthly county unemployment rate. Measures 1
through 4 are obtained from the 2018 American Community Survey: 5-Year Estimates
\citep{uscensusbureauAmericanCommunitySurvey2018}; unemployment rates are
gathered from the U.S. Bureau of Labor Statistics \citep{labor2020a}.

\hypertarget{analysis-and-results}{\section{Analysis and Results}\label{analysis-and-results}}

My analytic strategy proceeds in four steps. In Tables
\ref{tab:google-desc-table} (Stay-at-Home) and \ref{tab:vacc-desc-table}
(Vaccination Uptake), I present descriptive statistics for all study variables,
including variable ranges, means, and standard deviations across the two cases.
Then, in Tables \ref{tab:google-tab} - \ref{tab:vacc-tab}, I fit a series of
three linear mixed effects regression models using the nlme package in R
\citep{pinheiro_etal21, pinheiro_bates00} for our two cases, Stay-at-Home rates
and Vaccination rates. Linear mixed effects models are a form of hierarchical
linear models that contain both random and fixed effects. These models treat the
dependent variables as continuous. The following strategies I will outline are
identical for both case studies. Models 1 through 3 address hypotheses
\ref{hyp:infection} through \ref{hyp:discordance} respectively. Each model
utilizes normalized independent variables, i.e.\ variables that have been
centered and scaled to have a mean of 0 and standard deviation 1. The first
model is a baseline linear mixed effects model that employs all basic controls
to predict the dependent variable, allowing both time-varying and county-level
variables to predict the outcome. All variables have fixed effects, meaning that
the county-level exogenous effects are controlled for when estimating the
coefficient. In addition, models are specified with a random intercept per
county. The lme models are set with a autoregressive correlation structure
(\texttt{correlation\ =\ corAR1()}) to control for temporally autocorrelated
error structures; models are also optimized using Nelder--Mead, quasi--Newton and
conjugate--gradient algorithms for box-constrained optimization and simulated
annealing (\texttt{control\ =\ lmeControl(opt\ =\ ``optim'')}). This first model
will address Hypothesis \ref{hyp:infection}. To test Hypothesis
\ref{hyp:direction}, I estimate model 2 which builds on model 1 by first
introducing vaccination signal. Because the signal varies by week, I estimate
this model with an interaction between week and signal. And finally, because
signal may have divergent effects across counties, I set a random effect for
signal nested by county. A visual representation of this interaction can be seen
in Figures \ref{fig:plot-google-h2} and \ref{fig:plot-vacc-h2}. Model 3 further
elaborates on the previous model by introducing both a fixed effect for signal
discordance and an interaction term between signal and discordance. Figures
\ref{fig:plot-google-h3} and \ref{fig:plot-vacc-h3} depict just how signal and
discordance interact across these two cases.

\hypertarget{stay-at-home-rate-results}{
\subsection{Stay-at-Home Rate results}\label{stay-at-home-rate-results}}

\begin{table}[!ht]

\caption{\label{tab:google-tab}Linear Mixed Effects Regression Results for Stay-At-Home Rates}
\centering
\fontsize{8}{10}\selectfont
\begin{tabular}{lccc}
\toprule
  & Model 1 & Model 2 & Model 3\\
\midrule
Percent White & \num{0.088}*** & \num{0.058}** & \num{0.060}**\\
 & (\num{0.023}) & (\num{0.019}) & (\num{0.019})\\
Percent College Graduates & \num{-0.016} & \num{0.010} & \num{0.020}\\
 & (\num{0.028}) & (\num{0.023}) & (\num{0.024})\\
Percent over 65 & \num{-0.034}+ & \num{-0.031}* & \num{-0.037}*\\
 & (\num{0.018}) & (\num{0.015}) & (\num{0.015})\\
Median Income & \num{-0.042} & \num{-0.032} & \num{0.015}\\
 & (\num{0.030}) & (\num{0.024}) & (\num{0.025})\\
Monthly Unemployment Rate & \num{0.087}*** & \num{0.012}*** & \num{0.065}***\\
 & (\num{0.003}) & (\num{0.003}) & (\num{0.004})\\
Percent of GOP votes, 2016 & \num{0.077}* & \num{0.043} & \num{0.017}\\
 & (\num{0.032}) & (\num{0.026}) & (\num{0.027})\\
Percent Evangelical Christian & \num{-0.016} & \num{-0.008} & \num{-0.014}\\
 & (\num{0.022}) & (\num{0.018}) & (\num{0.019})\\
'Fox News' Trend & \num{-0.030}*** & \num{-0.028}*** & \num{-0.026}***\\
 & (\num{0.001}) & (\num{0.001}) & \vphantom{1} (\num{0.001})\\
'Social Distancing' Trend & \num{0.011}*** & \num{0.004}* & \num{0.001}\\
 & (\num{0.002}) & (\num{0.002}) & \vphantom{1} (\num{0.002})\\
'Covid Conspiracy' Trend & \num{0.024}*** & \num{0.006}*** & \num{0.010}***\\
 & (\num{0.002}) & (\num{0.002}) & (\num{0.002})\\
Covid Case Rate & \num{0.000} & \num{-0.009}* & \num{-0.007}+\\
 & (\num{0.004}) & (\num{0.004}) & (\num{0.004})\\
Week Number & \num{-0.002}+ & \num{-0.002}* & \num{-0.004}***\\
 & (\num{0.001}) & (\num{0.001}) & (\num{0.001})\\
Stay at Home Rate, county mean & \num{0.898}*** & \num{0.873}*** & \num{0.846}***\\
 & (\num{0.059}) & (\num{0.048}) & (\num{0.050})\\
Signal &  & \num{0.473}*** & \num{0.550}***\\
 &  & (\num{0.005}) & (\num{0.008})\\
Signal x Week &  & \num{-0.016}*** & \num{-0.018}***\\
 &  & (\num{0.000}) & (\num{0.000})\\
Signal Discordance &  &  & \num{-0.101}***\\
 &  &  & (\num{0.004})\\
Signal x Signal Discordance &  &  & \num{-0.092}***\\
 &  &  & (\num{0.002})\\
\midrule
Log.Lik. & \num{-23007.044} & \num{-19555.920} & \num{-18655.837}\\
\bottomrule
\multicolumn{4}{l}{\rule{0pt}{1em}N = 57,356, N of random Effects = 1375}\\
\multicolumn{4}{l}{\rule{0pt}{1em}* p $<$ .05. ** p $<$ .01. *** p $<$ .001 (two-tailed test).}\\
\multicolumn{4}{l}{\rule{0pt}{1em}Model 1 includes a random intercept for FIPS,}\\
\multicolumn{4}{l}{\rule{0pt}{1em}Models 2-3 include a random effect for Movement Signal by FIPS}\\
\end{tabular}
\end{table}


In Table \ref{tab:google-tab} and Figures \ref{fig:plot-google-h2} -
\ref{fig:plot-google-h3}, I present the elaboratory county-level models that
predict stay-at-home rates, or, time spent in residence. Model 1 addresses
Hypothesis \ref{hyp:infection}, that relatively higher local rates of infection
will lead to increased time spent in residence. This model's phi parameter is
0.933, which is a good indicator that adjacent time points for each county are
related and the model is specified correctly (Finch, Bolin, and Kelley 2014). In
this first model, many controls have an impact on the outcome. For instance,
when controlling for everything else, the percent of GOP votes in 2016, the
percentage of residents who identify racially as White, and higher rates of
unemployment increased the stay-at-home rate while Searches for norms online
seems to have an effect on the outcome: searches for `Fox News' are associated
with decreased stay-at-home rates, while searching for both `Social Distancing'
and `Covid Conspiracy' tend to increase time spent in residence. Interestingly,
the perceived threat of the virus, measured through COVID-19 case rates, were
not significantly related to stay-at-home rates in model 1. In other words, for
every 1 standard deviation increase in COVID-19 case rates, stay-at-home rates
actually decreased by 0.0003 (\emph{p} = 0.936), a very small and insignificant
effect. In this case, I fail to reject the null hypothesis that relatively
higher local rates of infection will lead to increased time spent in residence.

Table \ref{tab:google-tab} Model 2 investigates Hypothesis \ref{hyp:direction},
that increased average time spent in residence (signal direction) from alters
will have a positive effect on stay-at-home rates for the ego-county. This model
builds on model 1 by adding in the variable for Movement Signal (see
\protect\hyperlink{network-signal}{Network Signal} for specifications). For
every one standard deviation increase in the average time spent in residence of
the county alters, an ego tends to also increase its' own stay-at-home rate by
0.473 (\emph{p} \textless{} 0.001). I am therefore able to reject my null
hypothesis for Hypothesis \ref{hyp:direction}, that an increased average time
spent in residence (signal direction) from alters will have a positive effect on
time spent in residence for the ego-county.

\begin{figure}
{\centering \includegraphics[width=0.8\linewidth]{figs/paper3/plot-google-h2-1}}
\caption{Predicted Values of Stay-at-Home Rate by Movement Signal, Model 2}\label{fig:plot-google-h2}
\end{figure}

However, this coefficient is inadequate alone because the effect varies over
time; the interaction between week and movement signal is negative, meaning that
week partially moderates the effect of the signal. Figure
\ref{fig:plot-google-h2} illustrates this interaction. In the early days of the
COVID-19 pandemic, seeing a high rate of time spent in residence by alters led
to a high rate of staying-at-home for county egos. However, as the pandemic
progressed, these signals switched. In other words, towards the end of 2020,
other factors may have come into play and if a county saw its' alters social
distancing and spending time in residence, ego counties spent less time at home.
Theoretically, this may be because individuals saw their adjacent communities
with strong public health norms and felt safe and justified their own deviance.
However, if an ego county received signals that others were failing to social
distance, they were more likely to stay-at-home. In this case, the threat of the
virus may have been more evident for individuals.

\begin{figure}
{\centering \includegraphics[width=0.8\linewidth]{figs/paper3/plot-google-h3-1}}
\caption{Predicted Values of Stay-at-Home Rate Moderated by Discordance, Model 3}\label{fig:plot-google-h3}
\end{figure}

Model 3 then adds the concept of Signal Discordance to investigate Hypothesis
\ref{hyp:discordance}, that the effect of signal direction on stay-at-home rates
will be moderated by a diversity in signals (\emph{discordance}). Many of the
controls remain consistent throughout the models, with the exception of `Social
Distancing' Trend whose effect has been completely mediated by the inclusion of
signal discordance. Importantly, signal discordance has a negative effect on
stay-at-home rates, where every one standard deviation increase in discordance
lowers stay-at-home rates by -0.101 (\emph{p} \textless{} 0.001). The
interaction between signal and signal discordance have a coefficient of a
similar magnitude, with every one standard deviation increase in discordance and
signal resulting in -0.092 lower rates of time spent in residence, indicating
moderation. Figure \ref{fig:plot-google-h3} illustrates the moderating effect
that signal discordance has on signal. Under high discordance, the social
influence effect is almost completely moderated because there is no clear story
or norm forming. However, under low discordance, the signal is condensed,
allowing for social influence to have an effect. Explicitly, if a county is
receiving a wide array of low and high signals, their stay-at-home rates won't
be affected on average by social contagion. When a signal is concentrated or in agreement,
the theoretical effect of movement signaling on the ego are the strongest. With
this, I am able to reject the null hypothesis of no relationship for Hypothesis
\ref{hyp:discordance} and find that the effect of signal direction on time spent
in residence is moderated by diversity in signals.

\hypertarget{vaccination-rate-results}{\subsection{Vaccination Rate results}\label{vaccination-rate-results}}

\begin{table}[!h]

\caption{\label{tab:vacc-tab}Linear Mixed Effects Regression Results for Vaccination Rates}
\centering
\fontsize{8}{10}\selectfont
\begin{tabular}[t]{lccc}
\toprule
  & Model 1 & Model 2 & Model 3\\
\midrule
Percent White & \num{0.023}+ & \num{-0.029}*** & \num{-0.027}***\\
 & (\num{0.012}) & (\num{0.003}) & (\num{0.003})\\
Percent College Graduates & \num{-0.004} & \num{0.009}* & \num{0.010}*\\
 & (\num{0.013}) & (\num{0.004}) & (\num{0.004})\\
Percent over 65 & \num{-0.012} & \num{-0.002} & \num{0.000}\\
 & (\num{0.008}) & (\num{0.002}) & (\num{0.002})\\
Median Income & \num{0.002} & \num{-0.017}*** & \num{-0.016}***\\
 & (\num{0.011}) & (\num{0.003}) & (\num{0.003})\\
Monthly Unemployment Rate & \num{-0.031}*** & \num{-0.014}*** & \num{-0.013}***\\
 & (\num{0.001}) & (\num{0.001}) & (\num{0.001})\\
Percent of GOP votes, 2020 & \num{-0.070}*** & \num{0.031}*** & \num{0.028}***\\
 & (\num{0.014}) & (\num{0.004}) & (\num{0.004})\\
Percent Evangelical Christian & \num{0.000} & \num{0.000} & \num{-0.001}\\
 & (\num{0.009}) & (\num{0.003}) & (\num{0.003})\\
'Fox News' Trend & \num{0.002}*** & \num{0.000} & \num{0.000}*\\
 & (\num{0.000}) & (\num{0.000}) & \vphantom{4} (\num{0.000})\\
'Vaccine' Trend & \num{-0.001}*** & \num{0.000} & \num{0.000}+\\
 & (\num{0.000}) & (\num{0.000}) & \vphantom{3} (\num{0.000})\\
'Covid Conspiracy' Trend & \num{0.000}* & \num{0.001}*** & \num{0.001}***\\
 & (\num{0.000}) & (\num{0.000}) & \vphantom{2} (\num{0.000})\\
Covid Case Rate & \num{-0.008}*** & \num{0.001}*** & \num{0.001}***\\
 & (\num{0.000}) & (\num{0.000}) & \vphantom{1} (\num{0.000})\\
Week Number & \num{0.067}*** & \num{0.011}*** & \num{0.011}***\\
 & (\num{0.000}) & (\num{0.000}) & (\num{0.000})\\
Vaccination Rate, county mean & \num{0.800}*** & \num{0.201}*** & \num{0.170}***\\
 & (\num{0.018}) & (\num{0.005}) & (\num{0.005})\\
Vaccination Signal &  & \num{0.854}*** & \num{0.865}***\\
 &  & (\num{0.005}) & (\num{0.006})\\
Vaccination Signal x Week &  & \num{-0.005}*** & \num{-0.003}***\\
 &  & (\num{0.000}) & (\num{0.000})\\
Vaccination Discordance &  &  & \num{-0.072}***\\
 &  &  & \vphantom{1} (\num{0.002})\\
Vaccination Signal x Discordance &  &  & \num{-0.051}***\\
 &  &  & (\num{0.002})\\
\midrule
Log.Lik. & \num{118677.013} & \num{160000.978} & \num{160638.703}\\
\bottomrule
\multicolumn{4}{l}{\rule{0pt}{1em}N = 99,890, N of random Effects = 2819}\\
\multicolumn{4}{l}{\rule{0pt}{1em}* p $<$ .05. ** p $<$ .01. *** p $<$ .001 (two-tailed test).}\\
\multicolumn{4}{l}{\rule{0pt}{1em}Model 1 includes a random intercept for FIPS,}\\
\multicolumn{4}{l}{\rule{0pt}{1em}Models 2-3 include a random effect for Movement Signal by FIPS}\\
\end{tabular}
\end{table}

In Table \ref{tab:vacc-tab} and Figures \ref{fig:plot-vacc-h2} -
\ref{fig:plot-vacc-h3}, I present the elaboratory county-level models that
predict vaccination uptake rates. The parameter phi in model 1 is .985, which is
a good indicator that adjacent time points for each county are related and this
is an appropriate modeling technique for the data \citep{finch_etal14}. Table
\ref{tab:vacc-tab} Model 1 addresses Hypothesis \ref{hyp:infection}, that
relatively higher local rates of infection will lead to increased vaccination
uptake. In this first model, we see various controls impacting vaccination
rates. For instance, when controlling for everything else, higher income areas
lead to higher vaccination uptake. On the other hand, counties with higher
unemployment and higher votes for Donald J. Trump in the 2020 election are
associated with lower vaccination rates. Interestingly, counties that search for
Fox News are more likely to have higher vaccination rates while those performing
Google searches for information about the vaccine are likely to have lower rates
of vaccination. 
% TODO On the Fox News result--Interesting. In hindsight, this too may have to do with inhomogeneities within counties, e.g., maybe it's the conservatives in liberal counties that tune in avidly to Fox News.
The test of Hypothesis \ref{hyp:infection} also returns
surprising results: every 1 standard deviation increase in county COVID-19 case
rates is expected to yield a -0.008 decrease in vaccination rates. This may be
due to time ordering or reverse causality, where individuals in counties who
have higher rates of vaccination are less likely to take tests to COVID-19
infections because of the higher likelihood of asymptomaticity. However, it is
important to note that the Case Rate variable actually becomes distorted with
the addition of Vaccination Signal and Vaccination Discordance in models 2 and
3. Distortion occurs when the direction of a focal relationship reverses sign
once a third (distorter) variable is controlled, in this case, those related to
signal. Therefore, the results of Hypothesis \ref{hyp:infection} are quite
inconclusive.
% TODO Those who perform Google searches for vaccine info are likely to have lower vaccination rates --> Maybe they are searching for info about why the vaccine is bad !!

% TODO An increase in covid cases yields a decrease in vaccination rates. -- Well, the association would be like this if low vaccination rates lead to more covid cases.

\begin{figure}
{\centering \includegraphics[width=0.8\linewidth]{figs/paper3/plot-vacc-h2-1}}
\caption{Predicted Values of Vaccination Rate by Vaccination Signal, Model 2}\label{fig:plot-vacc-h2}
\end{figure}

Table \ref{tab:vacc-tab} model 2 tests the Hypothesis \ref{hyp:direction} that
increase vaccine uptake by alters will have a positive effect on vaccine uptake
for the ego-county. The results are incredibly clear: seeing alter counties
receiving vaccines is associated with higher vaccination rates. Specifically, a
1 standard deviation increase in vaccine signal, the average rate of vaccination
of a county's alters, is associated with a 0.854 percent increase of vaccination
rates for the ego county. The strength of this result leads me to reject the
null hypothesis for Hypothesis \ref{hyp:direction} and find that increased
average time spent in residence (signal direction) from alters will have a
positive effect on time spent in residence for the ego-county. The interaction
of vaccine signal can be seen in Figure \ref{fig:plot-vacc-h2}. While the
interaction is not as extreme as the results for Hypothesis \ref{hyp:direction}
for Stay-at-Home rates, there is still a distinction between high and low
vaccination signals over time. The slope for Higher Vaccination Signal is less
pronounced than Low Vaccination Signal. This is likely an artifact where some
counties vaccinated broadly early on, so their vaccination rate stays rather
constant in later weeks. However, overall, the message is clear: ego-counties
who `see' high levels of vaccinations among their alters have higher vaccination
rates; those who don't see that social norm forming are much less likely to
vaccinate.

\begin{figure}
{\centering \includegraphics[width=0.8\linewidth]{figs/paper3/plot-vacc-h3-1}}
\caption{Predicted Values of Vaccination Rate Moderated by Discordance, Model 3}\label{fig:plot-vacc-h3}
\end{figure}

Table \ref{tab:vacc-tab} Model 3 then models the moderation of signal direction
for vaccine uptake by diversity in signals (\emph{Discordance}). This is a
second test of Hypothesis \ref{hyp:discordance}, that the effect of signal
direction on vaccine uptake will be moderated by diversity in signals. The
coefficients in model 3 are largely consistent with model 2 and there is very
little change in the magnitude or significance of the coefficients. Vaccination
signal maintains to have a positive effect on the alter's vaccination rates,
while vaccination discordance has a negative effect on vaccination rates, where
every one standard deviation increase in discordance lowers rates by -0.072
(\emph{p} \textless{} 0.001). The interaction between signal and signal
discordance have a coefficient of a similar magnitude, with every one standard
deviation increase in discordance \emph{and} signal resulting in lower rates of
vaccination. Figure \ref{fig:plot-vacc-h3} illustrates this relationship, where
there is slight moderation of the effect on vaccine signal on vaccination rates.
If a county receives consistently high signals, they are more likely to have
more residents vaccinated. If a county receives high but inconsistent signals,
they are still decently likely to raise vaccination rates but to a lesser
extent. While not as strong of moderation the the findings for Stay-at-Home
Rates for Hypothesis \ref{hyp:discordance}, I also reject the null hypothesis
and find a moderation of the relationship between signal direction and vaccine
uptake by discordance.

\hypertarget{discussion-and-conclusion}{\section{Discussion and Conclusion}\label{discussion-and-conclusion}}

This paper examines how social norms are formed under conditions of uncertainty
using the case studies of stay-at-home rates and vaccination rates during the
COVID-19 pandemic. I test fear-based models of behavioral adaption to public
health recommendations as well as patterns of complex social contagion. This
article demonstrates that the establishment of social norms under patterns of
uncertainty in my two cases do follow the theorized framework of complex
contagion; however, I find a moderating effect of signal discordance to be an
overlooked factor in theories of contagion.

This study is sensitive to aggregation error and sampling error derived from the
multiple big-data sources \citep{facebook20, google2020}. The aggregation errors
are potentially linked to the assumption that the average stay-at-home behavior
of a county is a good representation of the diversity in behaviors within that
county. If stay-at-home rates are normally distributed, this is a correct
assumption to make; however, little is known about each county's underlying
distribution of behaviors that lead to the aggregate measures.
% TODO As perhaps illustrated hypothetically in my comment on p. 120

Despite limitations, this article is a unique and compelling contribution
to the academic study of social contagion and social norms. First, this article
demonstrates that it is possible to link, at least at a community level,
information search by individuals, social relationships, and observed health
behaviors without being limited to self-report measures of perceptions,
attitudes, or intentions about behavior. Sociological research has long debated
about the agency of individuals and the social and institutional structure in
which individuals are immersed. In this study, I aggregate beyond the individual
case to explore overall patterns of movement and vaccination which summarizes
individual agency of health behaviors together.

This article also provides a new adaption of the theory of complex contagion
using big data from the natural experiment of COVID-19. Because most studies of
complex contagion utilize agent-based modeling or studies of social media
\citep{aralExerciseContagionGlobal2017,sprague_house17,bond_etal12,latane_etal94},
combining measures of actual behavioral outcomes with data from social media
websites provides a further test of ecological and external validity for the
theory.

Researchers of contagion and social networks should be interested in this study.
My theory of discordance, or a diversity of signals received, is not found in
the literature on complex contagion; complex contagion focuses on the number of
reinforcing actors, while other theories focus on how different diffusants
interact with each other
\citep{houghton20,goldbergSocialContagionAssociative2018,mason_etal07}.
Discordance, on the other hand, looks specifically at the context of the
reinforcing actors among the other signals. This study finds that indeed the
context of the signals matters for diffusion.