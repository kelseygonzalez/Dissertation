% at least 150 words, generally no more than 500 words

This dissertation focuses on how people search for information and how people
rely on this information to inform their health-behaviors and develop social
norms. Information can be shared and promoted by others
through information campaigns and consumed by an individual, called information
push, or information can be intentionally sought out by individuals, 
called information pull, and their relation to the creation of social norms. 

The first article tests the construct validity of the Google Search Trends source,
an online indicator of computer-mediated information search. I attempt to
validate Google Search Trends for use as indicators of attitudes, disease prevalence
and political preferences using eight different validated data sources. %TODO double check number
I fail to find correlation among any of the Google Trends tested and their
validated indicators and show that there is no external validity of
Google Trends for these uses and social scientists will find no replacement for
high quality survey data with Google Trends. Instead, we must only use 
Google Trends to demonstrate interest or attention. 

Knowing that Google trends data only encompasses a small portion of the
information-seeking done by modern humans, my second article 
is motivated by the research question: How do computer-mediated or interpersonal 
information-seeking strategies vary across populations? Using original survey data
of 948 Americans, I investigate their experiences seeking out information about 
the Covid-19 vaccines among the following information seeking vehicles: 
personal connection, doctor, social networking site, online forum, and online search engine. 
I find little evidence that online search is more utilized than
seeking social support from personal network connections or health professionals
as I hypothesize based on uses and gratifications theory. I find that different
exposure points and information search vehicles hold real world consequences
through their associations with Covid-19 vaccination rates and intentions, as
information from a doctor increases the Covid-19 vaccination uptake while
receiving information from a Social Networking Site like Facebook or Twitter was
associated with lower odds of vaccination.

My final article takes a deeper dive into the formation of social norms
governing health behaviors in cases of extreme uncertainty using the cases of
both stay-at-home rates and vaccination rates as responses to public health recommendations to mitigate the Covid-19 pandemic. Using theories of associative diffusion and the integrated
theoretical framework of norms, I test models of behavioral adaption to public health
recommendations as well as patterns of complex social contagion using linear
mixed effects models. These models show that complex contagion is a valid 
framework for the social contagion of new norms during Covid-19. Importantly, I find
a novel moderating effect of signal discordance, the contextual diversity of
signals received by an ego. This paper shows that the contagion process is not
fully understood without looking at the context of each exposure point within the
range of exposures one experiences. 

The introductory chapter provides a summary of the research and an explanation of how this
research contributes to the sociology, social science, and society.
