% at least 150 words, generally no more than 500 words

The online world presents new avenues for the spread of information in addition to the traditional word of mouth effects of social networks. It also provides new big data sources for social science research because they are bigger, cheaper, and already available. In my dissertation, I combine these two areas of inquiry by examining the differences between the social network contagion of ideas (push) and information seeking (pull) and their relation to the creation of social norms. My first article tests the construct validity of the Google Search Trends source of big data as an indicator of three different cases, namely cultural attitudes, disease prevalence and voting behavior, using pairwise correlations . My second article establishes how individuals perform information search, whether through interpersonal networks or online, using my original survey data of 900 Adults in the United States. My final article builds on these by investigating how new health-behavior social norms form in the face of assortative signals through associative cognitive diffusion. This research will be a contribution to methodology of computational social science as it tests the reliability of a big data source, questions what behavioral processes lead to macro-level outcomes recorded in big data sources, and uses novel combinations of data sources and methodology to investigate questions of sociological interest. This research will also contribute to the sociology study of information technologies in the digital age, medical sociology, and social networks.