% at least 150 words, generally no more than 500 words

This dissertation focuses on how people search for information and how people
rely on this information to inform their health behaviors and develop social
norms. In academia and policy, the focus in research on information has most
often studied the information that is sent to consumers, seeing people as
passive receivers of information. This is otherwise known as information 'push'.
But, information is also intentionally sought out by individuals; called
information pull. My dissertation focuses on individuals as active agents in
their search for information and how the information discovered through these
push and pull processes relates to the creation of social norms.

My first article tests the criterion validity of Google Search Trends as an
indicator of computer-mediated information search. I attempt to validate Google
Search Trends for use as indicators of attitudes, disease prevalence and
political preferences using five different data sources. My analysis revealed no
correlation among any of the Google Trends tested and their validated
indicators. I demonstrate that there is no criterion validity of Google Trends
for the selected cases and social scientists will find no replacement for high
quality survey data with Google Trends. Instead, we must only use Google Trends
to demonstrate interest or attention.

Knowing that Google Trends data only encompasses a small portion of the
information-seeking done by modern humans, my second article is motivated by the
research question: How do computer-mediated or interpersonal information-seeking
strategies vary across populations? Using original survey data of 948 Americans,
I investigate their experiences seeking out information about COVID-19 vaccines.
I investigate five distinct information seeking modalities, or methods of
searching for information: personal connection, doctor, social networking site,
online forum, and online search engine. 
%I find little evidence that online search is more utilized than % seeking social support from personal network connections or health professionals % as I had hypothesized based on uses and gratifications theory, % that media users are not passive consumers but instead actively choose % media sources based on the satisfaction of their individual needs.
I find that different exposure points, the ways people first are exposed
to information without searching for it, and information search modalities hold
real world consequences through their associations with COVID-19 vaccination
intentions and rates. For example, I find that receiving or seeking out
information from a doctor increases COVID-19 vaccination uptake while receiving
information from a social networking site is associated with lower odds of
vaccination.

My final article takes a deeper dive into the formation of social norms
governing health behaviors in cases of extreme uncertainty. I specifically use
the cases of stay-at-home rates and vaccination rates as responses to public
health recommendations to mitigate the COVID-19 pandemic. Using the theories of
associative diffusion and the integrated theoretical framework of norms, I test
models of behavioral adaption to public health recommendations and patterns of
complex contagion (the need for repeated exposures to something novel for it
to diffuse) using linear mixed effects models. My results show that complex
contagion is a valid framework for the social contagion of new norms during
COVID-19. However, I find an important novel moderating effect of signal
discordance; if there is diversity in the information received by an ego,
contagion is less likely to occur. This paper shows that the contagion process
is not fully understood without looking at the context of each exposure to a
contagion within the range of contagions one experiences.

The introductory chapter provides a summary of the research and an explanation
of how this research contributes to sociology, social science, and society.
Namely, this paper provides important perspectives on search as an agentic
process and how the micro-level information seeking process of an individual can
lead to macro-level social norms. I show that information diffusion is disrupted
when conflicting information and behaviors are simultaneously diffusing and
therein contribute to research on diffusion, social networks, and social norms.


